\documentclass[areasetadvanced]{scrartcl}

\usepackage[utf8]{inputenc}
\usepackage[T2A]{fontenc}
\usepackage[english,russian]{babel}

\usepackage[footskip=1cm,left=25mm, right=15mm, top=20mm, bottom=20mm]{geometry}
\usepackage{setspace}
\usepackage{amsmath, amssymb} 
\usepackage{graphicx}
\usepackage{tikz}
\usetikzlibrary{arrows.meta}
\usepackage{float}
\usepackage{dashrule}
\usepackage{fancyhdr} 
\usepackage{hyperref} 
\usepackage{parskip}
\usepackage{textcomp, enumitem}
\usepackage{indentfirst}
\usepackage{graphicx}
\usepackage{algorithm}
\usepackage{algpseudocode}
\usepackage{array} 
\usepackage{geometry}
\usepackage{afterpage}
\usepackage{minted}
\setcounter{secnumdepth}{3} 
\setcounter{tocdepth}{3}    
\usepackage{listings} 

\newcommand{\icon}[1]{\includegraphics[height=1.2em]{#1}}

\tikzstyle{block} = [rectangle, rounded corners, minimum width=3cm, minimum height=1cm, text centered, draw=black, fill=lightgray]

\setkomafont{sectioning}{\normalfont\bfseries} 
\setkomafont{section}{\normalfont\Large\bfseries}
\setkomafont{subsection}{\normalfont\large\bfseries}
\setkomafont{subsubsection}{\normalfont\large\bfseries}
\setkomafont{paragraph}{\normalfont\large\bfseries} 

\lstset{
  language=Haskell,
  basicstyle=\ttfamily\small,
  keywordstyle=\color{blue}\bfseries,
  stringstyle=\color{red},
  commentstyle=\color{green!70!black},
  numbers=left,
  numberstyle=\tiny,
  stepnumber=1,
  numbersep=10pt,
  showstringspaces=false,
  breaklines=true,
  frame=single
}

\lstdefinelanguage{Lua}{
    keywords={function, end, if, then, else, elseif, for, while, do, repeat, until, break, return, local, and, or, not, true, false, nil},
    keywordstyle=\color{blue}\bfseries,
    stringstyle=\color{red},
    commentstyle=\color{green!70!black},
    morestring=[s]{"}{"},
    morestring=[s]{'}{'},
    morecomment=[l]{--},
    morecomment=[s]{--[[}{]]},
    basicstyle=\ttfamily\small,
    numbers=left,
    numberstyle=\tiny,
    stepnumber=1,
    numbersep=10pt,
    showstringspaces=false,
    breaklines=true,
    frame=single
}

\setlength{\parindent}{1.25cm}
\setcounter{tocdepth}{2}
\begin{document}
\sloppy
	\thispagestyle{empty}
	\begin{center}
		\large{МИНОБРНАУКИ РОССИИ} \par
		\vspace{0.3cm}
		\normalsize
		{ФЕДЕРАЛЬНОЕ ГОСУДАРСТВЕННОЕ АВТОНОМНОЕ ОБРАЗОВАТЕЛЬНОЕ УЧРЕЖДЕНИЕ ВЫСШЕГО ОБРАЗОВАНИЯ} \par
		\vspace{0.3cm}
		\textbf{\guillemotleft САНКТ-ПЕТЕРБУРГСКИЙ ПОЛИТЕХНИЧЕСКИЙ}
		\textbf{УНИВЕРСИТЕТ ПЕТРА ВЕЛИКОГО\guillemotright} \par
		\vspace{0.3cm}
		{Институт компьютерных наук и кибербезопасности}\par
		{Высшая школа технологий искусственного интеллекта}\par
	\end{center}
	\vfill
	\begin{center}

        \par
		{\Huge Лабораторная работа №2}\par
        \large {\guillemotleft LogiSim\guillemotright}\par
		\Large {Человеко-машинный интерфейс}
	\end{center}
	\vfill
	\begin{flushleft}
		Студент: \hspace{1.8cm} \rule[0pt]{2.5cm}{0.5pt}\hfill Салимли Айзек Мухтар Оглы\par
		\vspace{1.5cm}
		Преподаватель: \hspace{0.55cm} \rule[0pt]{2.5cm}{0.5pt}\hfill  Курочкин Михаил Александрович
	\end{flushleft}
	\vspace{0.5cm}
	\begin{flushright}
		\guillemotleft \rule[0pt]{0.8cm}{0.5pt}\guillemotright \rule[0pt]{2cm}{0.5pt} 20\rule[0pt]{0.5cm}{0.5pt} г.
	\end{flushright}
	\vfill
	\begin{center}
		Санкт-Петербург, 2025
	\end{center}
	\newpage
	\tableofcontents
	\newpage
	\newpage
	\section*{Введение}
	\addcontentsline{toc}{section}{Введение}
    Человеко-машинный интерфейс (ЧМИ) — это методы и средства обеспечения . непосредственного взаимодействия между оператором и технической системой, представляющих возможности оператору управлять этой системой и контролировать ее работу.

    В данном отчёте представлены результаты проектирования интерфейса программного обеспечения для поставленной задачи "Планирование меню для Великого поста".

    В ходе работы было проведено исследование, в рамках которого исследована предметная область, выявлены заинтересованные лица и их ожидания, проведён анализ пользователей программы, выявлены цели, задачи и пользовательские процессы.

    На основании исследования был спроектирован человеко-машинный интерфейс программы. В ходе проектирования разработана общая архитектура интерфейса, построена модель взаимодействия и структура команд. Разработанное решение детализировано до уровня экранной формы.

\end{document}