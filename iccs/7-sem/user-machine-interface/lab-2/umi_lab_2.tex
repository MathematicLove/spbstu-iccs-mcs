\documentclass[areasetadvanced]{scrartcl}

\usepackage[utf8]{inputenc}
\usepackage[T2A]{fontenc}
\usepackage[english,russian]{babel}

\usepackage[footskip=1cm,left=25mm, right=15mm, top=20mm, bottom=20mm]{geometry}
\usepackage{setspace}
\usepackage{amsmath, amssymb} 
\usepackage{graphicx}
\usepackage{tikz}
\usetikzlibrary{arrows.meta}
\usepackage{float}
\usepackage{dashrule}
\usepackage{fancyhdr} 
\usepackage{hyperref} 
\usepackage{parskip}
\usepackage{textcomp, enumitem}
\usepackage{indentfirst}
\usepackage{graphicx}
\usepackage{algorithm}
\usepackage{algpseudocode}
\usepackage{array} 
\usepackage{geometry}
\usepackage{afterpage}
\usepackage{minted}
\setcounter{secnumdepth}{3} 
\setcounter{tocdepth}{3}    
\usepackage{listings} 

\newcommand{\icon}[1]{\includegraphics[height=1.2em]{\#1}}

\tikzstyle{block} = [rectangle, rounded corners, minimum width=3cm, minimum height=1cm, text centered, draw=black, fill=lightgray]

\setkomafont{sectioning}{\normalfont\bfseries} 
\setkomafont{section}{\normalfont\Large\bfseries}
\setkomafont{subsection}{\normalfont\large\bfseries}
\setkomafont{subsubsection}{\normalfont\large\bfseries}
\setkomafont{paragraph}{\normalfont\large\bfseries} 

\lstset{
  language=Haskell,
  basicstyle=\ttfamily\small,
  keywordstyle=\color{blue}\bfseries,
  stringstyle=\color{red},
  commentstyle=\color{green!70!black},
  numbers=left,
  numberstyle=\tiny,
  stepnumber=1,
  numbersep=10pt,
  showstringspaces=false,
  breaklines=true,
  frame=single
}

\lstdefinelanguage{Lua}{
    keywords={function, end, if, then, else, elseif, for, while, do, repeat, until, break, return, local, and, or, not, true, false, nil},
    keywordstyle=\color{blue}\bfseries,
    stringstyle=\color{red},
    commentstyle=\color{green!70!black},
    morestring=[s]{"}{"},
    morestring=[s]{'}{'},
    morecomment=[l]{--},
    morecomment=[s]{--[[}{]]},
    basicstyle=\ttfamily\small,
    numbers=left,
    numberstyle=\tiny,
    stepnumber=1,
    numbersep=10pt,
    showstringspaces=false,
    breaklines=true,
    frame=single
}

\setlength{\parindent}{1.25cm}
\setcounter{tocdepth}{2}
\begin{document}
\sloppy
	\thispagestyle{empty}
	\begin{center}
		\large{МИНОБРНАУКИ РОССИИ} \par
		\vspace{0.3cm}
		\normalsize
		{ФЕДЕРАЛЬНОЕ ГОСУДАРСТВЕННОЕ АВТОНОМНОЕ ОБРАЗОВАТЕЛЬНОЕ УЧРЕЖДЕНИЕ ВЫСШЕГО ОБРАЗОВАНИЯ} \par
		\vspace{0.3cm}
		\textbf{\guillemotleft САНКТ-ПЕТЕРБУРГСКИЙ ПОЛИТЕХНИЧЕСКИЙ}
		\textbf{УНИВЕРСИТЕТ ПЕТРА ВЕЛИКОГО\guillemotright} \par
		\vspace{0.3cm}
		{Институт компьютерных наук и кибербезопасности}\par
		{Высшая школа технологий искусственного интеллекта}\par
	\end{center}
	\vfill
	\begin{center}

        \par
		{\Huge Лабораторная работа №2}\par
        \large {\guillemotleft LogiSim\guillemotright}\par
		\Large {Человеко-машинный интерфейс}
	\end{center}
	\vfill
	\begin{flushleft}
		Студент: \hspace{1.8cm} \rule[0pt]{2.5cm}{0.5pt}\hfill Салимли Айзек Мухтар Оглы\par
		\vspace{1.5cm}
		Преподаватель: \hspace{0.55cm} \rule[0pt]{2.5cm}{0.5pt}\hfill  Курочкин Михаил Александрович
	\end{flushleft}
	\vspace{0.5cm}
	\begin{flushright}
		\guillemotleft \rule[0pt]{0.8cm}{0.5pt}\guillemotright \rule[0pt]{2cm}{0.5pt} 20\rule[0pt]{0.5cm}{0.5pt} г.
	\end{flushright}
	\vfill
	\begin{center}
		Санкт-Петербург, 2025
	\end{center}
	\newpage
	\tableofcontents
\newpage
\section*{Введение}
\addcontentsline{toc}{section}{Введение}

	Человеко-машинный интерфейс (ЧМИ) — это методы и средства обеспечения непосредственного взаимодействия между оператором и технической системой, представляющих возможности оператору управлять этой системой и контролировать ее работу. Качество ЧМИ напрямую влияет на эффективность освоения программного обеспечения и продуктивность работы пользователя.
	
	В данном отчёте представлены результаты рецензирования человеко-машинного интерфейса программы Logisim (Java Edition for macOS) с позиции пользователя-новичка. Logisim представляет собой образовательный инструмент для проектирования и моделирования цифровых логических схем, широко используемый в учебном процессе.
	
	\textbf{Цели и задачи разработки ЧМИ}

	Основные цели проектирования человеко-машинного интерфейса включают:
	\begin{itemize}
		\item Обеспечение интуитивной понятности интерфейса для пользователей с разным уровнем подготовки
		\item Минимизацию времени обучения работе с программой
		\item Снижение количества ошибок при взаимодействии с системой
		\item Обеспечение эффективного выполнения целевых задач
		\item Создание комфортной рабочей среды, не вызывающей чрезмерной когнитивной нагрузки
	\end{itemize}

	\textbf{Основные проблемы разработки ЧМИ}

	При проектировании интерфейсов, особенно образовательного программного обеспечения, часто возникают следующие проблемы:
	\begin{itemize}
		\item Несоответствие ментальной модели новичка и концептуальной модели системы
		\item Избыточная сложность интерфейса для начального уровня освоения
		\item Непоследовательность в расположении элементов управления
		\item Неадаптивность интерфейса под разные сценарии использования
		\item Недостаточная обратная связь при совершении действий
	\end{itemize}

В ходе рецензирования был проведен анализ интерфейса Logisim с точки зрения соответствия принципам юзабилити, выявлены сильные и слабые стороны взаимодействия, а также сформулированы рекомендации по улучшению пользовательского опыта для начинающих пользователей.

\textbf{\textcolor{red}{Логические вентили используемые в отчете расписаны в \ref{section:prilo} главе.}}

\newpage
\section{Описание Стандарта ISO 9241}
Стандарт устанавливает руководство по проектированию пользовательских веб-интерфейсов, включая:
\begin{itemize}
\item Проектирование архитектуры и стратегии
\item Разработку информационного наполнения
\item Организацию навигации и поиска
\item Представление контента
\end{itemize}

\subsection{Основные принципы}
\begin{enumerate}
\item \textbf{Человеко-ориентированное проектирование}
\begin{itemize}
\item Соответствие интерфейса целям и задачам пользователей
\item Учет характеристик целевой аудитории
\item Обеспечение доступности для людей с ограниченными возможностями
\end{itemize}

\item \textbf{Разработка контента}
\begin{itemize}
\item Создание концептуальной модели информации
\item Независимость содержания от структуры и представления
\item Актуальность и полнота информации
\item Предоставление текстовых альтернатив для мультимедиа
\end{itemize}

\item \textbf{Навигация и поиск}
\begin{itemize}
\item Интуитивно понятная навигационная структура
\item Минимизация количества шагов для доступа к информации
\item Предоставление карты сайта и навигационных цепочек
\item Функции простого и расширенного поиска
\end{itemize}

\item \textbf{Представление информации}
\begin{itemize}
\item Последовательный визуальный дизайн
\item Четкое визуальное разделение элементов
\item Испознаваемость ссылок и интерактивных элементов
\item Адаптивность к различным устройствам
\end{itemize}

\item \textbf{Общие требования}
\begin{itemize}
\item Обеспечение конфиденциальности данных
\item Поддержка многоязычных интерфейсов
\item Устойчивость к ошибкам пользователей
\item Приемлемое время загрузки страниц
\end{itemize}
\end{enumerate}

\subsection{Соответствие стандарту}
Для подтверждения соответствия рекомендуется:
\begin{itemize}
\item Проводить оценку применимости каждого требования
\item Документировать выполненные рекомендации
\item Использовать проверочные вопросники
\end{itemize}

\newpage
\section{Анализ человеко-машинного интерфейса Logisim}

\subsection{Цель приложения}
Проектирование и моделирование цифровых логических схем, обеспечивающего эффективное обучение основам цифровой логики и компьютерной архитектуры для студентов и преподавателей.

\subsection{Заинтересованные лица}
\begin{itemize}
\item \textbf{Студенты} - основная целевая аудитория, изучающая цифровую логику и компьютерную архитектуру
\item \textbf{Преподаватели} - используют приложение в учебном процессе для демонстрации и проведения практических работ
\end{itemize}

\subsection{Ожидания заинтересованных лиц}
\begin{itemize}
\item \textbf{Студенты:}
  \begin{itemize}
  \item Быстрое освоение интерфейса без предварительной подготовки
  \item Интуитивно понятный процесс создания и тестирования схем
  \item Возможность самостоятельной работы с минимальным руководством
  \item Низкий порог входа для понимания основных принципов
  \end{itemize}
  
\item \textbf{Преподаватели:}
  \begin{itemize}
  \item Надежность и стабильность работы во время занятий
  \item Возможность быстрого создания демонстрационных материалов
  \item Простота проверки студенческих работ
  \end{itemize}
\end{itemize}

\subsection{Планируемые результаты использования продукта}
\begin{itemize}
\item \textbf{Образовательные результаты:}
  \begin{itemize}
  \item Освоение принципов цифровой логики студентами
  \item Формирование практических навыков проектирования логических схем
  \item Снижение когнитивной нагрузки при изучении концепций логических схем
  \item Визуализация процесса работы логических схем 
  \end{itemize}

\item \textbf{Операционные результаты:}
  \begin{itemize}
  \item Сокращение времени на подготовку к практическим занятиям
  \item Упрощение процесса создания учебных материалов
  \item Повышение эффективности учебного процесса
  \item Снижение потребности в дорогостоящем лабораторном оборудовании
  \end{itemize}

\item \textbf{Технические результаты:}
  \begin{itemize}
  \item Кросс-платформенная доступность (JVM)
  \item Стабильная работа на стандартных компьютерных конфигурациях
  \end{itemize}
\end{itemize}

\newpage
\section{Ментальная модель пользователя «новичка»}

Ментальная модель пользователя-новичка, впервые знакомящегося с Logisim, формируется на основе следующих ключевых аспектов:

\subsection{Исходные знания и ожидания}
\begin{itemize}
\item \textbf{Базовые теоретические знания}: Предпологается, что пользователь знаком с фундаментальными логическими операциями (AND, OR, NOT, XOR) на концептуальном уровне в рамках учебной программы
\item \textbf{Практический опыт}: Возможное отсутствие или наличие ограниченного опыта взаимодействия с симуляторами логических схем
\item \textbf{Ожидание интуитивного интерфейса}: Предположение о простоте освоения программного обеспечения благодаря визуальному представлению элементов
\item \textbf{Образовательный контекст использования}: Восприятие программного обеспечения как инструмента учебного назначения, а не профессиональной системы
\end{itemize}

\subsection{Когнитивные процессы и стратегии взаимодействия}
\begin{itemize}
\item \textbf{Ориентация на визуальные элементы}: Преимущественное использование графических подсказок и визуальной организации интерфейса
\item \textbf{Поиск знакомых элементов интерфейса}: Активный поиск инструментов, аналогичных известным графическим редакторам
\item \textbf{Поэтапное освоение функциональности}: Предпочтение последовательного перехода от базовых к сложным функциям системы
\item \textbf{Потребность в оперативной обратной связи}: Ожидание визуального подтверждения корректности выполняемых действий
\end{itemize}

\subsection{Типовые модели взаимодействия с системой}
\begin{itemize}
\item \textbf{Конструкторский подход}: Представление процесса проектирования как компоновки стандартных логических элементов
\item \textbf{Графико-ориентированное восприятие}: Интерпретация интерфейса как средства визуального проектирования со специализированными компонентами
\item \textbf{Ассоциация с лабораторным оборудованием}: Проведение аналогий с физическими лабораторными работами по сборке электронных схем
\item \textbf{Линейный рабочий процесс}: Ожидание последовательного перехода от этапа проектирования к этапу тестирования схемы
\end{itemize}

\subsection{Критические аспекты освоения системы}
\begin{itemize}
\item \textbf{Соответствие теоретическим знаниям}: Понимание практической реализации теоретических принципов логических операций
\item \textbf{Принцип функционирования симулятора}: Разграничение режимов проектирования и выполнения схемы
\item \textbf{Методология построения схем}: Освоение последовательности операций от выбора компонентов до их соединения и верификации
\item \textbf{Визуализация состояний системы}: Интерпретация цветового кодирования и отображения сигналов в схеме
\end{itemize}

\subsection{Требования к пользовательскому интерфейсу}
\begin{itemize}
\item \textbf{Минимальный порог вхождения}: Возможность начать работу без предварительного глубокого изучения документации
\item \textbf{Интуитивная понятность функций}: Очевидность назначения инструментов и элементов управления
\item \textbf{Предотвращение ошибочных действий}: Защита от критических ошибок и обеспечение возможности их исправления
\item \textbf{Образовательная поддержка}: Наличие примеров реализации и контекстных подсказок, способствующих освоению системы
\end{itemize}

\newpage
\section{Задачи решаемые пользователем}
Приложение Logisim обеспечивает решение следующих ключевых задач для пользователя-новичка:

\subsection{Базовые задачи проектирования логических схем}
\begin{itemize}
\item \textbf{Создание элементарных логических схем}: Построение базовых комбинационных схем с использованием логических элементов, светодиодов, и др.
\item \textbf{Визуализация работы логических элементов}: Наблюдение за распространением сигналов через схему с цветовым кодированием состояний
\item \textbf{Соединение компонентов}: Освоение методики корректного соединения входов и выходов логических элементов
\end{itemize}

\subsection{Задачи анализа и отладки}
\begin{itemize}
\item \textbf{Диагностика ошибок соединений}: Выявление и исправление некорректных подключений в схеме
\item \textbf{Анализ временных диаграмм}: Наблюдение за изменением сигналов во времени при помощи встроенного инструмента анализа
\item \textbf{Тестирование функциональности}: Проверка корректности работы схемы при различных входных условиях
\end{itemize}

\subsection{Образовательные задачи}
\begin{itemize}
\item \textbf{Понимание булевой алгебры}: Практическое применение законов и теорем булевой алгебры при проектировании схем
\item \textbf{Изучение комбинационных схем}: Освоение принципов построения сумматоров, мультиплексоров, декодеров
\item \textbf{Знакомство с последовательностными схемами}: Исследование работы триггеров, регистров и счетчиков
\item \textbf{Освоение иерархического проектирования}: Создание сложных схем на основе ранее разработанных модулей
\end{itemize}

\subsection{Практические учебные задачи}
\begin{itemize}
\item \textbf{Выполнение лабораторных работ}: Реализация типовых учебных заданий по проектированию цифровых устройств
\item \textbf{Подготовка к экзаменационным заданиям}: Тренировка решения задач по проектированию логических схем
\item \textbf{Создание учебных проектов}: Разработка завершенных функциональных устройств на основе изученного материала
\item \textbf{Сравнительный анализ решений}: Оценка различных вариантов реализации одной и той же логической функции
\end{itemize}

\newpage 
\section{Описание задач и процессов}

\subsection{Метафоры и идиомы проектирования}

\begin{itemize}
\item \textbf{Метафора «электронной лаборатории»}: 
\begin{itemize}
\item Процесс проектирования представлен как работа в виртуальной лаборатории с радиодеталями
\item Пользователь «берёт» компоненты из «панели инструментов» и «размещает» на «монтажном поле»
\item Соединение элементов осуществляется «проводами», аналогично пайке в реальной схеме
\end{itemize}

\item \textbf{Идиома «конструктора логических схем»}:
\begin{itemize}
\item Библиотека компонентов организована по принципу конструктора с типовыми элементами
\item Возможность «вкладывать» созданные схемы в качестве новых компонентов (иерархическое проектирование)
\item «Подсветка» активных соединений при наведении курсора
\end{itemize}

\item \textbf{Метафора «цифрового осциллографа»}:
\begin{itemize}
\item Встроенный анализатор схем работает по принципу осциллографа, отображая состояния сигналов
\item «Зонд» в виде курсора позволяет «измерять» значения в разных точках схемы
\item Цветовая индикация сигналов (красный - 0, зелёный - 1, синий - ошибка)
\end{itemize}

\item \textbf{Идиома «интерактивного моделирования»}:
\begin{itemize}
\item Режим «ручного управления» позволяет изменять входные сигналы щелчком мыши
\item «Тактовый генератор» обеспечивает пошаговое выполнение для анализа последовательностных схем
\item «Пауза» и «продолжение» симуляции аналогичны управлению реальным устройством
\end{itemize}
\end{itemize}

\subsection{Процессы решения типовых задач}

\begin{itemize}
\item \textbf{Процесс создания простой логической схемы}:
\begin{itemize}
\item Выбор компонентов из палитры инструментов (логические элементы, входы/выходы)
\item Размещение элементов на рабочем поле перетаскиванием
\item Соединение компонентов проводами методом «точка-точка»
\item Тестирование схемы переключением входных значений
\end{itemize}

\item \textbf{Процесс отладки и анализа}:
\begin{itemize}
\item Запуск симуляции для наблюдения за распространением сигналов
\item Использование инструмента «Зонд» для проверки значений в узлах схемы
\item Применение «Таблицы истинности» для верификации работы схемы
\item Пошаговая симуляция для анализа временных характеристик
\end{itemize}

\item \textbf{Процесс иерархического проектирования}:
\begin{itemize}
\item Создание подсхемы из группы логических элементов
\item Определение интерфейса подсхемы (входы и выходы)
\item Использование созданной подсхемы как макрокомпонента в основной схеме
\item Многоуровневое вложение подсхем для сложных проектов
\end{itemize}

\item \textbf{Процесс работы с памятью и последовательностными схемами}:
\begin{itemize}
\item Настройка тактовой частоты для синхронных элементов
\item Инициализация регистров и памяти начальными значениями
\item Наблюдение за изменением состояний в течение нескольких тактов
\item Анализ диаграмм состояний с помощью встроенных инструментов
\end{itemize}
\end{itemize}

\subsection{Идиомы взаимодействия с интерфейсом}

\begin{itemize}
\item \textbf{«Выбор-и-размещение»}: Стандартная идиома для добавления компонентов - выбор инструмента и щелчок в месте размещения
\item \textbf{«Перетаскивание проводов»}: Создание соединений протягиванием линии между точками подключения
\item \textbf{«Щелчок-для-изменения»}: Изменение значений входных контактов простым щелчком мыши
\item \textbf{«Контекстное меню»}: Доступ к свойствам компонентов через правый щелчок мыши
\item \textbf{«Панель атрибутов»}: Настройка параметров компонентов через отдельную панель свойств
\end{itemize}

\newpage
\section{Описание возможностей LogiSim}

\subsection{Действия для создания простой логической схемы}

\begin{itemize}
\item \textbf{Библиотека компонентов}:
\begin{itemize}
\item Встроенная палитра логических элементов (AND, OR, NOT, XOR, NAND, NOR)
\item Инструменты ввода/вывода (контакты, светодиоды, семисегментные индикаторы)
\item Размещение компонентов методом drag-and-drop на рабочую область
\end{itemize}

\item \textbf{Функции соединения элементов}:
\begin{itemize}
\item Инструмент «Провод» для соединения точек подключения
\item Автоматическое выравнивание проводов по сетке
\item Визуальная индикация корректных соединений
\item Возможность разрыва и перенаправления проводов
\end{itemize}

\item \textbf{Управление параметрами компонентов}:
\begin{itemize}
\item Панель атрибутов для настройки свойств каждого элемента
\item Изменение количества входов у логических элементов
\item Настройка меток и описаний компонентов
\item Возможность поворота и отражения элементов
\end{itemize}
\end{itemize}

\subsection{Действия отладки и анализа}

\begin{itemize}
\item \textbf{Инструменты симуляции}:
\begin{itemize}
\item Кнопки управления симуляцией (Запуск, Пауза, Продолжить, Сброс)
\item Пошаговое выполнение для анализа последовательностных схем
\item Режим ручного управления входными сигналами
\end{itemize}

\item \textbf{Действия диагностики}:
\begin{itemize}
\item Инструмент «Зонд» для измерения значений в точках схемы
\item Цветовая индикация состояний сигналов (0, 1, ошибка, высокий импеданс)
\item Визуализация распространения сигналов в реальном времени
\item Обнаружение конфликтов и коротких замыканий
\end{itemize}

\item \textbf{Аналитические инструменты}:
\begin{itemize}
\item Встроенный генератор таблиц истинности
\item Лог изменений состояний схемы
\item Графическое отображение временных диаграмм
\item Статистика использования компонентов
\end{itemize}
\end{itemize}

\subsection{Действия иерархического проектирования}

\begin{itemize}
\item \textbf{Создание подсхем}:
\begin{itemize}
\item Функция «Создать подсхему» для группировки элементов
\item Определение интерфейса подсхемы (входные и выходные порты)
\item Возможность многократного использования созданных подсхем
\end{itemize}

\item \textbf{Управление иерархией}:
\begin{itemize}
\item Навигация по уровням вложенности схем
\item Древовидное представление структуры проекта
\item Быстрый доступ к определению подсхемы
\end{itemize}

\item \textbf{Библиотека пользовательских компонентов}:
\begin{itemize}
\item Сохранение часто используемых подсхем в библиотеку
\item Импорт/экспорт пользовательских компонентов
\item Совместное использование подсхем между проектами
\end{itemize}
\end{itemize}

\subsection{Работа с памятью и последовательностными схемами}

\begin{itemize}
\item \textbf{Компоненты памяти}:
\begin{itemize}
\item Триггеры различных типов (D, T, JK, RS)
\item Регистры и сдвиговые регистры
\item Память ROM и RAM с настраиваемой разрядностью
\item Счетчики и таймеры
\end{itemize}

\item \textbf{Управление тактированием}:
\begin{itemize}
\item Настраиваемые генераторы тактовых импульсов
\item Синхронизация нескольких тактовых доменов
\item Возможность ручного тактирования
\end{itemize}

\item \textbf{Функции инициализации}:
\begin{itemize}
\item Установка начальных состояний для элементов памяти
\item Загрузка данных в память из файла
\item Сброс схемы в начальное состояние
\end{itemize}

\item \textbf{Анализ последовательностных процессов}:
\begin{itemize}
\item Отслеживание изменений состояний во времени
\item Визуализация работы конечных автоматов
\item Анализ временных параметров схемы
\end{itemize}
\end{itemize}

\subsection{Вспомогательные возможности}

\begin{itemize}
\item \textbf{Документирование}:
\begin{itemize}
\item Текстовые аннотации и комментарии
\item Графические элементы для оформления схем
\item Экспорт схем в различные форматы (PNG, PDF)
\end{itemize}

\item \textbf{Управление проектом}:
\begin{itemize}
\item Система файлов для хранения схем
\item Автосохранение и восстановление проектов
\item Управление версиями схем
\end{itemize}

\item \textbf{Настройка интерфейса}:
\begin{itemize}
\item Изменение масштаба отображения схемы
\item Настройка цветовых схем и отображения
\item Персонализация панелей инструментов
\end{itemize}
\end{itemize}

\newpage
\section{Описание структуры диалога и его обоснование}

\subsection{Общая архитектура диалоговой системы}

Интерфейс Logisim реализует комбинированную модель диалога, сочетающую элементы непосредственного управления и меню-ориентированного взаимодействия.

\subsection{Структура диалога для основных процессов}

\begin{itemize}
\item \textbf{Диалог создания новой схемы}:
\begin{itemize}
\item Инициация: Файл $\rightarrow$ Создать
\item Параметры: имя схемы, размер холста
\item Подтверждение: кнопка «Создать»
\item Обоснование: соответствие принципу управляемости
\end{itemize}

\item \textbf{Диалог добавления компонентов}:
\begin{itemize}
\item Инициация: выбор компонента из палитры инструментов
\item Визуальная обратная связь: изменение курсора
\item Размещение: щелчок на рабочей области
\item Обоснование: принцип соответствия ожиданиям пользователя
\end{itemize}

\item \textbf{Диалог настройки компонентов}:
\begin{itemize}
\item Инициация: двойной щелчок на компоненте
\item Структура: панель атрибутов с параметрами
\item Подтверждение: автоматическое применение изменений
\item Обоснование: минимализм диалога
\end{itemize}

\item \textbf{Диалог соединения элементов}:
\begin{itemize}
\item Инициация: выбор инструмента «Провод»
\item Визуальный отклик: подсветка точек подключения
\item Процесс: протягивание линии
\item Завершение: индикация установленного соединения
\item Обоснование: принцип понятности
\end{itemize}
\end{itemize}

\subsection{Диалоги управления симуляцией}

\begin{itemize}
\item \textbf{Запуск симуляции}:
\begin{itemize}
\item Инициация: кнопка «Симуляция» $\rightarrow$ «Запуск»
\item Визуальный отклик: анимация сигналов
\item Контроль: панель управления симуляцией
\item Обоснование: соответствие принципу управляемости
\end{itemize}

\item \textbf{Диалог отладки}:
\begin{itemize}
\item Инициация: активация инструмента «Зонд»
\item Взаимодействие: щелчок по точкам схемы
\item Отображение: всплывающие подсказки
\item Обоснование: принцип самодокументируемости
\end{itemize}

\item \textbf{Диалог анализа временных диаграмм}:
\begin{itemize}
\item Инициация: Окно $\rightarrow$ Анализатор схем
\item Структура: отдельное окно с графиком
\item Управление: настройка диапазона и масштаба
\item Обоснование: разделение ответственности
\end{itemize}
\end{itemize}

\subsection{Иерархические диалоги}

\begin{itemize}
\item \textbf{Создание подсхемы}:
\begin{itemize}
\item Инициация: Проект $\rightarrow$ Добавить подсхему
\item Диалог определения: задание имени и портов
\item Визуализация: создание символа подсхемы
\item Обоснование: поддержка концептуальной модели
\end{itemize}

\item \textbf{Навигация по иерархии}:
\begin{itemize}
\item Панель навигации: древовидное представление
\item Переход: двойной щелчок для перехода
\item Ориентация: отображение текущего положения
\item Обоснование: принцип информативности
\end{itemize}
\end{itemize}

\subsection{Диалоги управления проектом}

\begin{itemize}
\item \textbf{Диалог сохранения}:
\begin{itemize}
\item Стандартный диалог сохранения файлов
\item Дополнительные параметры: версия формата
\item Обоснование: соответствие ожиданиям платформы
\end{itemize}

\item \textbf{Диалог экспорта}:
\begin{itemize}
\item Выбор формата: PNG, PDF
\item Настройки качества и размера
\item Предпросмотр результата
\item Обоснование: принцип гибкости
\end{itemize}
\end{itemize}

\subsection{Обоснование выбранной структуры диалога}

Структура диалога в Logisim основана на следующих принципах:

\begin{itemize}
\item Соответствие задачам пользователя
\item Минимизация когнитивной нагрузки
\item Обеспечение обратной связи
\item Поддержка навигации
\item Доступность для начинающих пользователей
\end{itemize}

\newpage
\section{Описание ошибок пользователей}

\subsection{Ошибки управления проектом}

\begin{itemize}
\item \textbf{Потеря данных при закрытии проекта}:
\begin{itemize}
\item Закрытие программы без сохранения текущей схемы
\item Случайное нажатие "Не сохранять" в диалоговом окне
\item Отсутствие привычки регулярного сохранения работы
\end{itemize}

\item \textbf{Некорректное именование файлов}:
\begin{itemize}
\item Использование запрещенных символов в именах проектов
\item Сохранение в неподдерживаемых форматах
\item Потеря связи между основным файлом и подсхемами
\end{itemize}
\end{itemize}

\subsection{Ошибки проектирования схем}

\begin{itemize}
\item \textbf{Некорректное соединение компонентов}:
\begin{itemize}
\item Соединение выходов нескольких элементов без использования буфера
\item Пересечение проводов без создания узла соединения
\item Неправильная ориентация компонентов при размещении
\end{itemize}

\item \textbf{Ошибки разрядности сигналов}:
\begin{itemize}
\item Соединение компонентов с разной разрядностью шин
\item Несоответствие количества входов и выходов в подсхемах
\item Неправильная настройка битности мультиплексоров и демультиплексоров
\end{itemize}

\item \textbf{Проблемы с тактированием}:
\begin{itemize}
\item Отсутствие тактового сигнала в последовательностных схемах
\item Неправильное подключение сброса (reset) триггеров
\item Создание гонок сигналов (race conditions) в комбинационных схемах
\end{itemize}
\end{itemize}

\subsection{Ошибки навигации и ориентации}
\begin{itemize}
\item \textbf{Неправильное использование инструментов}:
\begin{itemize}
\item Путаница между режимом редактирования и режимом симуляции
\item Непонимание разницы между инструментами выделения и проводника
\item Случайная активация не тех инструментов палитры
\end{itemize}
\end{itemize}

\subsection{Ошибки симуляции и отладки}

\begin{itemize}
\item \textbf{Неправильная интерпретация состояний}:
\begin{itemize}
\item Непонимание цветового кодирования сигналов (красный/зеленый/синий)
\item Путаница между высоким импедансом и неопределенным состоянием
\item Неверная трактовка результатов в анализаторе схем
\end{itemize}

\item \textbf{Ошибки конфигурации симуляции}:
\begin{itemize}
\item Установка неподходящей тактовой частоты для конкретной схемы
\item Неправильная настройка параметров запуска симуляции
\item Отсутствие проверки начальных условий перед запуском
\end{itemize}
\end{itemize}


\subsection{Ошибки работы с интерфейсом}

\begin{itemize}
\item \textbf{Неэффективное использование пространства}:
\begin{itemize}
\item Создание слишком плотных или слишком разреженных схем
\item Неоптимальное расположение компонентов на рабочем поле
\end{itemize}

\item \textbf{Ошибки настройки параметров}:
\begin{itemize}
\item Неправильная установка задержек распространения сигналов
\item Ошибочная конфигурация параметров компонентов через панель атрибутов
\item Использование несовместимых настроек для связанных элементов
\end{itemize}
\end{itemize}

Данная классификация ошибок позволяет выявить слабые места в интерфейсе Logisim и разработать меры по их предотвращению, улучшая общий пользовательский опыт для начинающих пользователей.

\newpage
\section{Выбор методов обработки ошибок пользователя}

\subsection{Для ошибки управления проектом}

\begin{itemize}
\item \textbf{Автосохранение с интервалом 5 минут} и напоминание о сохранении при закрытии
\item \textbf{Валидация имени файла} при вводе с подсветкой недопустимых символов
\end{itemize}

\subsection{Для ошибки проектирования схем}

\begin{itemize}
\item \textbf{Автоматическая установка буферов} при соединении выходов нескольких элементов
\item \textbf{Визуальная индикация несовместимости разрядности} цветом соединений
\end{itemize}

\subsection{Для ошибки навигации и ориентации}

\begin{itemize}
\item \textbf{Четкое визуальное различие режимов} редактирования и симуляции через цветовую схему
\item \textbf{Подсветка активного инструмента} на панели с текстовым описанием
\end{itemize}

\subsection{Для ошибки симуляции и отладки}

\begin{itemize}
\item \textbf{Легенда цветового кодирования} в углу экрана с пояснением состояний
\item \textbf{Автоматическая проверка готовности схемы} к симуляции перед запуском
\end{itemize}

\subsection{Для ошибки работы с интерфейсом}

\begin{itemize}
\item \textbf{Функция автоматического упорядочивания} компонентов по сетке
\item \textbf{Валидация параметров в реальном времени} с подсказками допустимых значений
\end{itemize}

\newpage
\section{Описание структуры команд пользователей}

\subsection{Организация системы команд}

Система команд Logisim организована по функциональному принципу и включает следующие основные группы:

\subsection{Команды управления проектом}
\begin{itemize}
\item \textbf{Создание/открытие/сохранение}: 
\begin{itemize}
\item Файл $\rightarrow$ Новый проект (Ctrl-N)
\item Файл $\rightarrow$ Открыть (Ctrl-O) 
\item Файл $\rightarrow$ Сохранить (Ctrl-S)
\end{itemize}

\item \textbf{Экспорт результатов}:
\begin{itemize}
\item Файл $\rightarrow$ Экспорт изображения
\item Файл $\rightarrow$ Печать схемы (Ctrl-P)
\end{itemize}
\end{itemize}

\subsection{Команды редактирования схем}
\begin{itemize}
\item \textbf{Базовые операции}:
\begin{itemize}
\item Правка $\rightarrow$ Отменить (Ctrl-Z)
\item Правка $\rightarrow$ Повторить (Ctrl-Y)
\item Правка $\rightarrow$ Удалить (Delete)
\end{itemize}

\item \textbf{Манипуляции с компонентами}:
\begin{itemize}
\item Правка $\rightarrow$ Копировать (Ctrl-C)
\item Правка $\rightarrow$ Вставить (Ctrl-V)
\item Правка $\rightarrow$ Повернуть/Отразить
\end{itemize}
\end{itemize}

\subsection{Команды проектирования}
\begin{itemize}
\item \textbf{Добавление компонентов}:
\begin{itemize}
\item Выбор из палитры инструментов (логические элементы, провода, входы/выходы)
\item Перетаскивание на рабочую область
\end{itemize}

\item \textbf{Настройка компонентов}:
\begin{itemize}
\item Двойной щелчок $\rightarrow$ Свойства компонента
\item Правка $\rightarrow$ Атрибуты (для изменения параметров)
\end{itemize}
\end{itemize}

\subsection{Команды симуляции}
\begin{itemize}
\item \textbf{Управление выполнением}:
\begin{itemize}
\item Симуляция $\rightarrow$ Запуск (Ctrl-R)
\item Симуляция $\rightarrow$ Пауза (Ctrl-P)
\item Симуляция $\rightarrow$ Сброс (Ctrl-T)
\end{itemize}

\item \textbf{Инструменты анализа}:
\begin{itemize}
\item Симуляция $\rightarrow$ Анализатор схем
\item Инструмент $\rightarrow$ Зонд (для проверки сигналов)
\end{itemize}
\end{itemize}

\subsection{Команды навигации}
\begin{itemize}
\item \textbf{Просмотр схем}:
\begin{itemize}
\item Вид $\rightarrow$ Увеличить (Ctrl-Plus)
\item Вид $\rightarrow$ Уменьшить (Ctrl-Minus)
\item Вид $\rightarrow$ Показать сетку
\end{itemize}

\item \textbf{Организация проекта}:
\begin{itemize}
\item Проект $\rightarrow$ Добавить подсхему
\item Проект $\rightarrow$ Переименовать схему
\end{itemize}
\end{itemize}

\subsection{Структура доступа к командам}

\begin{itemize}
\item \textbf{Главное меню}: Иерархическая организация всех команд
\item \textbf{Панели инструментов}: Быстрый доступ к часто используемым командам
\item \textbf{Контекстные меню}: Команды, доступные по правому щелчку
\item \textbf{Горячие клавиши}: Ускорение работы для повторяющихся операций
\end{itemize}

Данная структура команд обеспечивает логичную организацию взаимодействия пользователя с системой, соответствуя принципам последовательности и доступности.

\newpage
\section{Физическая реализация команд пользователей}
\subsubsection{Манипуляция мышью}
\begin{itemize}
\item \textbf{Левый щелчок}:
\begin{itemize}
\item Выбор и активация инструментов на панели
\item Размещение компонентов на рабочей области
\item Изменение значений входных контактов
\item Активация кнопок управления симуляцией
\end{itemize}

\item \textbf{Правый щелчок}:
\begin{itemize}
\item Открытие контекстного меню компонентов
\item Доступ к свойствам и атрибутам элементов
\item Быстрый доступ к часто используемым операциям
\end{itemize}

\item \textbf{Перетаскивание (Drag-and-Drop)}:
\begin{itemize}
\item Перемещение компонентов по рабочей области
\item Изменение размеров и положения элементов
\item Создание проводов между точками подключения
\end{itemize}
\end{itemize}

\subsubsection{Клавиатурный ввод}
\begin{itemize}
\item \textbf{Базовые комбинации клавиш}:
\begin{itemize}
\item Ctrl-S - сохранение проекта
\item Ctrl-Z - отмена последнего действия
\item Ctrl-Y - повтор отмененного действия
\item Delete - удаление выделенных элементов
\end{itemize}

\item \textbf{Навигационные клавиши}:
\begin{itemize}
\item Стрелки - точное позиционирование компонентов
\item Page Up/Down - масштабирование схемы
\item Space - временное переключение инструментов
\end{itemize}
\end{itemize}


\newpage
\section{Описание структуры информационной модели экранной формы}
\subsection{Главное окно программы (начальное состояние)}
\begin{itemize}
\item \textbf{Заголовок окна}: Отображает название программы и текущий проект ("Logisim: main of Untitled")
\item \textbf{Панель главного меню}: Горизонтальное меню с разделами File, Edit, Project, Simulate, Window, Help
\item \textbf{Панель палитры компонентов}: Вертикальная панель слева с категориями элементов:
  \begin{itemize}
  \item Wiring (Соединения)
  \item Gates (Логические элементы)
  \item Plexers (Мультиплексоры)
  \item Arithmetic (Арифметические)
  \item Memory (Память)
  \item Input/Output (Ввод/Вывод)
  \item Base (Базовые)
  \end{itemize}
\item \textbf{Рабочая область}: Центральная часть окна с точечной сеткой для размещения компонентов
\item \textbf{Область навигации по проекту}: Левая верхняя часть показывает структуру проекта 
\item \textbf{Строка состояния}: Нижняя часть окна с информацией о масштабе отображения (100\%)
\end{itemize}

Ниже на рисунке \ref{fig:homepage} представлен графический интерфейс главного окна:

\begin{figure}[H]
	\centering
	\includegraphics[width=0.68\linewidth]{img/homepage.png}
	\caption{Главное окно LogiSim}
	\label{fig:homepage}
\end{figure}

\textbf{Визуальная организация информации}:
\begin{itemize}
\item Четкое разделение на функциональные зоны: навигация, инструменты, рабочая область
\item Иерархическое представление палитры компонентов по функциональным группам
\item Использование точечной сетки для облегчения позиционирования элементов
\item Минималистичный дизайн, не перегруженный элементами управления
\end{itemize}

\subsection{Добавление логического элемента AND Gate}
\begin{itemize}
\item \textbf{Контекст операции}: Пользователь выбрал элемент "AND Gate" из категории "Gates" в палитре компонентов
\item \textbf{Панель атрибутов компонента}: Правая панель отображает настраиваемые параметры выбранного элемента:
  \begin{itemize}
  \item Facing (Ориентация): East (Восток)
  \item Data Bits (Разрядность данных): 1 бит
  \item Gate Size (Размер элемента): Medium (Средний)
  \item Number Of Inputs (Количество входов): 5
  \item Output Value (Выходное значение): 0/1
  \item Label (Метка): пусто
  \item Label Font (Шрифт метки): SansSerif Plain...
  \item Negate 1-3 (Инверсия входов): No (Нет)
  \end{itemize}
\item \textbf{Визуальное представление элемента}: В палитре компонентов элемент "AND Gate" выделен, указывая на текущий выбор
\item \textbf{Инструмент активен}: В строке состояния или интерфейсе указано "Tool: AND Gate", подтверждая выбор инструмента
\item \textbf{Готовность к размещению}: Курсор изменился на указатель размещения элемента на рабочей области
\end{itemize}

На рисунке \ref{fig:gateAnd} представлено добавление вентиля (логического элемента) "И" ("AND"): 

\begin{figure}[H]
	\centering
	\includegraphics[width=0.68\linewidth]{img/gateAnd.png}
	\caption{Вентиль И в LogiSim}
	\label{fig:gateAnd}
\end{figure}


\textbf{Организация интерфейса настройки}:
\begin{itemize}
\item Группировка связанных параметров (основные свойства, настройки метки, параметры входов)
\item Использование выпадающих списков для параметров с ограниченным набором значений
\item Четкое разделение между конфигурируемыми атрибутами и информационными полями
\item Предварительный просмотр конфигурации элемента через панель атрибутов
\end{itemize}

\subsection{Меню категории компонентов Wiring}
\begin{itemize}
\item \textbf{Контекст навигации}: Раскрытое меню категории "Wiring" (Соединения) в палитре компонентов
\item \textbf{Структура проекта}: В верхней части отображается иерархия текущего проекта:
  \begin{itemize}
  \item Untitled* - название проекта (звездочка указывает на несохраненные изменения)
  \item main - основная схема проекта
  \end{itemize}
\item \textbf{Элементы категории Wiring}: Список компонентов для соединений и проводки:
  \begin{itemize}
  \item Splitter (Разветвитель)
  \item Pin (Контакт)
  \item Probe (Щуп)
  \item Tunnel (Туннель)
  \item Pull Resistor (Подтягивающий резистор)
  \item Clock (Тактовый генератор)
  \item Constant (Константа)
  \item Power (Питание)
  \end{itemize}
\item \textbf{Визуальная организация}: Древовидная структура с отступами, показывающая отношения между элементами
\item \textbf{Навигационный паттерн}: Позволяет одновременно видеть структуру проекта и доступные компоненты выбранной категории
\end{itemize}

На рисунке \ref{fig:menu}, представлено меню для раздела Wiring: 

\begin{figure}[H]
	\centering
	\includegraphics[width=0.48\linewidth]{img/menu.png}
	\caption{Меню LogiSim}
	\label{fig:menu}
\end{figure}

\textbf{Особенности интерфейса}:
\begin{itemize}
\item Компактное представление двухуровневой информации (проект - компоненты)
\item Использование символа "*" для индикации несохраненного состояния
\item Логическая группировка связанных компонентов проводки в одной категории
\item Прокручиваемый список для работы с большим количеством элементов
\end{itemize}

\subsection{Эксперемент}

Для проведения эксперемента, было необходимо выбрать тактовую частоту, как изображено на рисунке \ref{fig:simulation}:

\begin{figure}[H]
	\centering
	\includegraphics[width=0.48\linewidth]{img/simulation.png}
	\caption{Настройка симуляции}
	\label{fig:simulation}
\end{figure}

После найстройки тактов, приведем небольшой пример, реализующий решение следующей булевой функции:

$$
	F_{2}(x,y,z,w) = \overline{(\overline{x} \land \overline{y}) \oplus \overline{(z \land w)}}, x = 0, y = 0, z = 0, w = 0.
$$

$$
	F_{2}(x,y,z,w) = \overline{(1 \land 1) \oplus \overline{(0 \land 0)}} = \overline{0} = 1
$$

Индекатор - должен светиться. Верно. 

\begin{itemize}
	\item \textbf{Входные устройства}:
	\begin{itemize}
	\item 4 кнопки (Button) - источники входных сигналов
	\end{itemize}
	
	\item \textbf{Логические элементы}:
	\begin{itemize}
	\item 3 инвертора (NOT) - выполняют логическое отрицание
	\item 2 элемента И (AND) - выполняют логическое умножение
	\item 1 элемент XOR - выполняет исключающее ИЛИ
	\item 1 финальный инвертор (NOT) - инвертирует выходной сигнал
	\end{itemize}
	
	\item \textbf{Выходное устройство}:
	\begin{itemize}
	\item 1 светодиод (LED) - визуальный индикатор результата
	\end{itemize}
	\end{itemize}
	
	\textbf{Суть эксперимента}: Исследование работы комбинационной логической схемы, построенной на базовых логических элементах с различными типами вентилей. Схема демонстрирует каскадное соединение элементов для реализации сложной логической функции.

	\textbf{Реализация}: Для реализации схемотичного решения заданной булевой функции (смотреть рисунок \ref{fig:Experiment}), необходимо:
	\begin{itemize}
		\item Разместить входные переменные в рабочем пространстве: \begin{itemize}
			\item Зайти в раздел "Wiring" в меню и выбрать элемент "Button"
			\item Перетащить элемент "Button" в рабочее пространство по клику мыши
			\item Повторить эти действия для всех входных переменных (x, y, z, w)
		\end{itemize}
		\item Разместить логические элементы в рабочем пространстве: \begin{itemize}
			\item Зайти в раздел "Gates" в меню и выбрать элемент "NOT"
			\item Перетащить элемент "NOT" в рабочее пространство по клику мыши
			\item Повторить эти действия для всех логических элементов
		\end{itemize}
		\item Разместить выходное устройство в рабочем пространстве: \begin{itemize}
			\item Зайти в раздел "Wiring" в меню и выбрать элемент "LED"
			\item Перетащить элемент "LED" в рабочее пространство по клику мыши
		\end{itemize}
		\item Соединить элементы проводами (желательно без пересечений проводов): \begin{itemize}
		\item Соединенить элементы проводами можно компьютерной "мышью" по точкам подключения элементов
		\item Желательно ссылаться на ГОСТ 2.701-84 «Схемы. Виды и типы. Общие требования к выполнению»
		\item При необходимости пересечения проводов, можно использовать элемент "Tunnel"
		\item Провода могут иметь ломаную линию.
		\end{itemize}
	\end{itemize}

	\textbf{Комментарий: } \textcolor{red}{Построение логической схемы, не вызвало особых трудностей, инстерфейс предсказуем и удобен. Стоит отметить что при неверном подключении программа выдает ошибку о неправильной реализации схемы. Соединение логических элементов, может вызвать трудности при проектировании прямых проводов, однако это играет лишь визуальную роль. }
	
	На рисунке \ref{fig:Experiment}, изображено решение булевой функции:

	\begin{figure}[H]
		\centering
		\includegraphics[width=0.68\linewidth]{img/Experiment.png}
		\caption{Эксперемент}
		\label{fig:Experiment}
	\end{figure}

\subsection{Создание новой схемы}
\begin{itemize}
\item \textbf{Контекст}: Диалоговое окно создания новой схемы (Input Circuit Name)
\item \textbf{Поле ввода}:
\begin{itemize}
\item Circuit Name: - текстовое поле для ввода названия новой схемы
\end{itemize}
\item \textbf{Кнопки управления}:
\begin{itemize}
\item Cancel - отмена создания схемы
\item OK - подтверждение создания схемы с введенным названием
\end{itemize}
\end{itemize}

\begin{figure}[H]
	\centering
	\includegraphics[width=0.68\linewidth]{img/newCiruit.png}
	\caption{Диалог создания новой схемы}
	\label{fig:newCiruit}
\end{figure}

\textbf{Особенности интерфейса}:
\begin{itemize}
\item Минималистичный диалог с фокусом на вводе имени схемы
\item Стандартное расположение кнопок подтверждения/отмены
\item Отсутствие дополнительных настроек при создании схемы
\end{itemize}

\newpage
\section{Описание дизайна экранной формы (шрифтов, цветовой палитры)}

\subsection{Типографика и шрифты}

\textbf{Основной шрифт интерфейса}:
\begin{itemize}
\item \textbf{Семейство}: SansSerif (без засечек)
\item \textbf{Начертание}: Plain (обычное)
\item \textbf{Размер}: 12 пунктов
\item \textbf{Применение}: метки компонентов, текст интерфейса
\end{itemize}

\textbf{Шрифт заголовков}:
\begin{itemize}
\item \textbf{Семейство}: SansSerif
\item \textbf{Начертание}: Bold (полужирное)
\item \textbf{Применение}: заголовки окон, названия категорий
\end{itemize}

\textbf{Шрифт древовидной структуры}:
\begin{itemize}
\item \textbf{Семейство}: SansSerif
\item \textbf{Размер}: 12 пунктов
\item \textbf{Особенности}: моноширинное начертание для элементов дерева проектов
\end{itemize}

\subsection{Цветовая палитра интерфейса}

\textbf{Основные цвета фона}:
\begin{itemize}
\item {Основной фон: \#F0F0F0, R:240, G:240, B:240}
\item {Фон областей ввода: \#FFFFFF, R:255, G:255, B:255}
\item {Фон выделенных элементов: \#E1E1E1, R:225, G:225, B:225}
\end{itemize}

\textbf{Цвета текста}:
\begin{itemize}
\item \textcolor[RGB]{0,0,0}{Основной текст: \#000000, R:0, G:0, B:0}
\item \textcolor[RGB]{64,64,64}{Вторичный текст: \#404040, R:64, G:64, B:64}
\item \textcolor[RGB]{128,128,128}{Неактивные элементы: \#808080, R:128, G:128, B:128}
\end{itemize}

\textbf{Акцентные цвета}:
\begin{itemize}
\item \textcolor[RGB]{0,102,204}{Акцентный синий: \#0066CC, R:0, G:102, B:204}
\item \textcolor[RGB]{0,128,0}{Зеленый индикатор: \#008000, R:0, G:128, B:0}
\item \textcolor[RGB]{204,0,0}{Красный индикатор: \#CC0000, R:204, G:0, B:0}
\end{itemize}

\textbf{Цвета границ и разделителей}:
\begin{itemize}
\item \textcolor[RGB]{192,192,192}{Границы элементов: \#C0C0C0, R:192, G:192, B:192}
\item \textcolor[RGB]{160,160,160}{Разделители: \#A0A0A0, R:160, G:160, B:160}
\end{itemize}

\subsection{Композиция и расположение элементов}

\textbf{Общая структура}:
\begin{itemize}
\item Трехпанельный layout с древовидной навигацией слева
\item Область свойств и параметров справа
\item Основная рабочая область по центру
\item Строка состояния в нижней части окна
\end{itemize}

\textbf{Навигационная панель}:
\begin{itemize}
\item Иерархическое представление проекта с отступами
\item Группировка компонентов по функциональным категориям
\item Подсветка активной схемы (main)
\end{itemize}

\textbf{Панель свойств}:
\begin{itemize}
\item Табличное представление параметров
\item Четкое разделение на свойства и их значения
\item Единообразное выравнивание элементов формы
\end{itemize}

\newpage
\section{ИСО 9241-161. Элементы графического пользовательского интерфейса}

\subsection{Анализ соответствия интерфейса Logisim стандарту ИСО 9241-161}

\begin{table}[H]
\centering
\caption{Анализ соответствия интерфейса Logisim стандарту ИСО 9241-161}
\begin{tabular}{|p{0.3\textwidth}|p{0.3\textwidth}|p{0.3\textwidth}|}
\hline
\textbf{Требование стандарта} & \textbf{Реализация в Logisim} & \textbf{Соответствие} \\
\hline
\hline
\textbf{Консистентность шрифтов} & \textcolor{green}{-} Полное соответствие & \textcolor{green}{Полное} \\
 Единый стиль шрифтов во всем интерфейсе &  SansSerif Plain 12 для всех элементов & \\
 Соответствие размеров шрифтов &  Размер 12pt соблюдается везде & \\
\hline
\textbf{Цветовая консистентность} & \textcolor{green}{-} Частичное соответствие & \textcolor{orange}{Частичное} \\
 Единая цветовая схема &  Серая палитра с синими акцентами & \\
 Достаточный контраст &  Контраст черного на сером соответствует & \\
\hline
\textbf{Группировка элементов} & \textcolor{green}{-} Полное соответствие & \textcolor{green}{Полное} \\
 Логическая группировка связанных элементов &  Древовидная структура категорий & \\
 Визуальное разделение групп &  Отступы и иерархия & \\
\hline
\textbf{Навигационная ясность} & \textcolor{green}{-} Полное соответствие & \textcolor{green}{Полное} \\
 Понятная иерархия навигации &  Четкая структура проекта и компонентов & \\
 Индикация текущего положения &  Выделение активной схемы "main" & \\
\hline
\textbf{Состояния элементов} & \textcolor{red}{Х} Неполное соответствие & \textcolor{red}{Неполное} \\
 Визуальная индикация состояний &  Только звездочка для несохраненных изменений & \\
 Feedback пользовательских действий &  Ограниченная визуальная обратная связь & \\
\hline
\end{tabular}
\end{table}

\subsection{Детальный анализ элементов интерфейса}

\textbf{Типографика по стандарту:}
\begin{itemize}
\item \textcolor{green}{-} \textbf{Размер шрифта}: 12pt соответствует минимальным требованиям читаемости
\item \textcolor{green}{-} \textbf{Семейство шрифтов}: SansSerif рекомендуется для интерфейсов
\item \textcolor{orange}{-} \textbf{Межстрочный интервал}: Соответствует базовым требованиям, но мог бы быть улучшен
\end{itemize}

\textbf{Цветовая схема:}
\begin{itemize}
\item \textcolor{green}{-} \textbf{Контрастность}: Соотношение контраста ~4.5:1 (черный на \#F0F0F0)
\item \textcolor{orange}{-} \textbf{Цветовая кодировка}: Ограниченное использование цвета для передачи информации
\item \textcolor{green}{-} \textbf{Единообразие}: Соблюдение единой палитры во всех элементах
\end{itemize}

\textbf{Компоновка и структура:}
\begin{itemize}
\item \textcolor{green}{-} \textbf{Выравнивание}: Строгое соблюдение сетки и отступов
\item \textcolor{green}{-} \textbf{Группировка}: Логическая группировка по функциональности
\item \textcolor{green}{-} \textbf{Иерархия}: Четкая визуальная иерархия элементов
\end{itemize}

\textbf{Навигация и ориентация:}
\begin{itemize}
\item \textcolor{green}{-} \textbf{Ориентация в структуре}: Понятное расположение элементов навигации
\item \textcolor{orange}{-} \textbf{Индикация выбора}: Минимальная визуальная индикация активных элементов
\item \textcolor{green}{-} \textbf{Маркировка состояний}: Использование "*" для несохраненных изменений
\end{itemize}

\newpage
\section{Экспертное заключение}
Интерфейс Logisim демонстрирует хорошее соответствие базовым требованиям ИСО 9241-161 в области типографики и структурной организации, но требует улучшений в области интерактивности и визуальной обратной связи.
Так же, разработчики интерфейса Logisim предпологают, что пользователь уже ознакомлен с базовыми принципами булевой логики и названиям элементов логических схем.

\newpage
\section*{Выводы}
\addcontentsline{toc}{section}{Выводы}
Проведённый анализ человеко-машинного интерфейса программы Logisim с позиции пользователя-новичка позволил выявить как сильные стороны программного обеспечения, так и направления для потенциального улучшения.

\textbf{Сильные стороны интерфейса}:
\begin{itemize}
\item Интерфейс демонстрирует высокую степень соответствия принципам логической организации элементов управления
\item Структура меню и палитры компонентов интуитивно понятна и соответствует ментальной модели пользователя-новичка
\item Цветовая схема и типографика обеспечивают достаточную контрастность и читаемость элементов управления
\item Система визуальной обратной связи при построении схем способствует пониманию принципов цифровой логики
\end{itemize}

\textbf{Области для улучшения}:
\begin{itemize}
\item Отмечается недостаточная визуальная индикация состояний элементов и ограниченная обратная связь при пользовательских действиях
\item Интерфейс предполагает наличие у пользователя базовых знаний булевой алгебры, что может создавать барьеры для абсолютных новичков
\item Отсутствует адаптивная система подсказок, которая могла бы упростить начальное освоение программы
\item Навигация по сложным проектам с множеством подсхем может вызывать затруднения у неопытных пользователей
\end{itemize}

\textbf{Общая оценка}:

Logisim представляет собой эффективный образовательный инструмент, интерфейс которого в целом соответствует требованиям стандарта ИСО 9241-161. Программа успешно решает поставленные образовательные задачи, обеспечивая наглядное представление принципов работы цифровых логических схем. Однако для повышения доступности и снижения порога входа для пользователей без предварительной подготовки рекомендуется реализовать дополнительные средства визуальной поддержки и контекстной помощи.

\newpage
\addcontentsline{toc}{section}{Список литературы}
\begin{thebibliography}{9}
    \bibitem{cooper}
    Алан Купер. \emph{Об интерфейсе. Основы проектирования взаимодействия}. --- 2-е изд. --- М.: ИМВ, 2009. — 688 с.
    \end{thebibliography}

	\newpage 
	\section*{Приложение А}
	\addcontentsline{toc}{section}{Приложение А}
\label{section:prilo}
\begin{itemize}
	\item NOT - Логическое \guillemetleft Не\guillemetright, $\neg$, $\overline{x}$ 
	\item AND - Логическое \guillemetleft И\guillemetright, $\land$, $\&$
	\item OR - Логическое \guillemetleft Или\guillemetright, $\lor$, $\|\|$
	\item XOR - \guillemetleft Исключающее Или\guillemetright, $\oplus$.
\end{itemize}

\begin{tabular}{|c|c||c|c|c|c|}
	\hline
	A & B & NOT A & A AND B & A OR B & A XOR B \\
	\hline
	0 & 0 & 1 & 0 & 0 & 0 \\
	0 & 1 & 1 & 0 & 1 & 1 \\
	1 & 0 & 0 & 0 & 1 & 1 \\
	1 & 1 & 0 & 1 & 1 & 0 \\
	\hline
	\end{tabular}

	\begin{figure}[H]
		\centering
		\includegraphics[width=0.48\linewidth]{img/image copy 4.png}
	\end{figure}
\end{document}