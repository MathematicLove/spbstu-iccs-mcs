\documentclass[areasetadvanced]{scrartcl}

\usepackage[utf8]{inputenc}
\usepackage[T2A]{fontenc}
\usepackage[english,russian]{babel}

\usepackage[footskip=1cm,left=25mm, right=15mm, top=20mm, bottom=20mm]{geometry}
\usepackage{setspace}
\usepackage{amsmath, amssymb} 
\usepackage{graphicx}
\usepackage{tikz}
\usetikzlibrary{arrows.meta}
\usepackage{float}
\usepackage{dashrule}
\usepackage{fancyhdr} 
\usepackage{hyperref} 
\usepackage{parskip}
\usepackage{textcomp, enumitem}
\usepackage{indentfirst}
\usepackage{graphicx}
\usepackage{algorithm}
\usepackage{algpseudocode}
\usepackage{array} 
\usepackage{geometry}
\usepackage{afterpage}
\usepackage{minted}
\setcounter{secnumdepth}{3} 
\setcounter{tocdepth}{3}    
\usepackage{listings} 

\newcommand{\icon}[1]{\includegraphics[height=1.2em]{\#1}}

\tikzstyle{block} = [rectangle, rounded corners, minimum width=3cm, minimum height=1cm, text centered, draw=black, fill=lightgray]

\setkomafont{sectioning}{\normalfont\bfseries} 
\setkomafont{section}{\normalfont\Large\bfseries}
\setkomafont{subsection}{\normalfont\large\bfseries}
\setkomafont{subsubsection}{\normalfont\large\bfseries}
\setkomafont{paragraph}{\normalfont\large\bfseries} 

\lstset{
  language=Haskell,
  basicstyle=\ttfamily\small,
  keywordstyle=\color{blue}\bfseries,
  stringstyle=\color{red},
  commentstyle=\color{green!70!black},
  numbers=left,
  numberstyle=\tiny,
  stepnumber=1,
  numbersep=10pt,
  showstringspaces=false,
  breaklines=true,
  frame=single
}

\lstdefinelanguage{Lua}{
    keywords={function, end, if, then, else, elseif, for, while, do, repeat, until, break, return, local, and, or, not, true, false, nil},
    keywordstyle=\color{blue}\bfseries,
    stringstyle=\color{red},
    commentstyle=\color{green!70!black},
    morestring=[s]{"}{"},
    morestring=[s]{'}{'},
    morecomment=[l]{--},
    morecomment=[s]{--[[}{]]},
    basicstyle=\ttfamily\small,
    numbers=left,
    numberstyle=\tiny,
    stepnumber=1,
    numbersep=10pt,
    showstringspaces=false,
    breaklines=true,
    frame=single
}

\setlength{\parindent}{1.25cm}
\setcounter{tocdepth}{2}
\begin{document}
\sloppy
	\thispagestyle{empty}
	\begin{center}
		\large{МИНОБРНАУКИ РОССИИ} \par
		\vspace{0.3cm}
		\normalsize
		{ФЕДЕРАЛЬНОЕ ГОСУДАРСТВЕННОЕ АВТОНОМНОЕ ОБРАЗОВАТЕЛЬНОЕ УЧРЕЖДЕНИЕ ВЫСШЕГО ОБРАЗОВАНИЯ} \par
		\vspace{0.3cm}
		\textbf{\guillemotleft САНКТ-ПЕТЕРБУРГСКИЙ ПОЛИТЕХНИЧЕСКИЙ}
		\textbf{УНИВЕРСИТЕТ ПЕТРА ВЕЛИКОГО\guillemotright} \par
		\vspace{0.3cm}
		{Институт компьютерных наук и кибербезопасности}\par
		{Высшая школа технологий искусственного интеллекта}\par
	\end{center}
	\vfill
	\begin{center}

        \par
		{\Huge ДОКЛАД}\par
        \large {\guillemotleft Обучающие системы\guillemotright}\par
		\Large {Человеко-машинный интерфейс}
	\end{center}
	\vfill
	\begin{flushleft}
		Студент: \hspace{1.8cm} \rule[0pt]{2.5cm}{0.5pt}\hfill Салимли Айзек Мухтар Оглы\par
		\vspace{1.5cm}
		Преподаватель: \hspace{0.55cm} \rule[0pt]{2.5cm}{0.5pt}\hfill  Курочкин Михаил Александрович
	\end{flushleft}
	\vspace{0.5cm}
	\begin{flushright}
		\guillemotleft \rule[0pt]{0.8cm}{0.5pt}\guillemotright \rule[0pt]{2cm}{0.5pt} 20\rule[0pt]{0.5cm}{0.5pt} г.
	\end{flushright}
	\vfill
	\begin{center}
		Санкт-Петербург, 2025
	\end{center}
	\newpage
	\tableofcontents

\newpage
\section*{Введение}
\addcontentsline{toc}{section}{Введение}

Человеко-машинный интерфейс (ЧМИ) - это методы и средства обеспечения непосредственного взаимодействия между оператором и технической системой, представляющих возможности оператору управлять этой системой и контролировать ее работу. Качество ЧМИ напрямую влияет на эффективность освоения программного обеспечения и продуктивность работы пользователя. Однако, в контексте высокорисковых и наукоемких областей, таких как авиация, атомная энергетика, медицина или сложные технологические процессы, эта эффективность трансформируется в требования к безопасности, надежности и безошибочности действий оператора. Именно здесь на первый план выходят специализированные обучающие системы - \textbf{тренажеры} (симуляторы).

Тренажер представляет собой комплекс аппаратно-программных средств, максимально достоверно имитирующий реальную рабочую среду, объект управления (например, кабину самолета, пульт управления энергоблоком, хирургический комплекс) и условия его функционирования. Основная цель тренажера - обеспечить формирование и отработку у обучаемых устойчивых профессиональных навыков, моторных и когнитивных моделей поведения в штатных, нештатных и аварийных ситуациях \textbf{без риска} для людей, оборудования и окружающей среды. Таким образом, тренажер является закономерным и высшим проявлением принципов человеко-машинного взаимодействия, где интерфейс перестает быть просто инструментом, а становится центральным элементом искусственно созданной, но максимально приближенной к реальности, обучающей экосистемы.

Обучающие системы на базе тренажеров можно рассматривать как подкласс человеко-машинных интерфейсов, для которых характерны следующие отличительные черты:
\begin{itemize}
\item \textbf{Высокая степень физического и функционального подобия (фидельность)}: Интерфейс тренажера стремится к максимальному соответствию реальному оборудованию по тактильным, визуальным и звуковым характеристикам.
\item \textbf{Обратная связь, ориентированная на анализ действий}: Система не только реагирует на команды оператора, но и записывает, анализирует и оценивает каждое его действие, предоставляя развернутый отчет и рекомендации по итогам тренировки.
\item \textbf{Возможность моделирования редких и опасных ситуаций}: Тренажер позволяет создавать и повторять сценарии, которые в реальности встречаются крайне редко или являются нежелательными для отработки «в живую».
\item \textbf{Адаптивность учебного процесса}: Современные системы могут подстраивать сложность сценариев в реальном времени в зависимости от успеваемости обучаемого, выступая в роли «интеллектуального инструктора».
\end{itemize}

Развитие технологий виртуальной (VR) и дополненной (AR) реальности, систем силовой обратной связи (haptic feedback) и сложного математического моделирования физических процессов расширяет границы применения тренажеров, делая их ключевым инструментом не только в профессиональной подготовке, но и в образовании, промышленности и научных исследованиях. Поэтому анализ принципов построения человеко-машинного интерфейса обучающих систем является актуальной задачей, лежащей на стыке эргономики, педагогики и информационных технологий.

\newpage
\section{Классификация и архитектура тренажерных систем}

Тренажерные системы, как подкласс обучающих человеко-машинных интерфейсов, представляют собой сложные комплексы, чья структура и состав напрямую определяются целями обучения и требуемым уровнем подготовки операторов. Анализ существующих систем позволяет выявить устойчивые классификационные признаки и архитектурные решения, регламентируемые в том числе национальными стандартами.

\subsection{Классификация тренажерных систем}

Классификация тренажеров может быть проведена по нескольким ключевым признакам, отражающим их дидактические, технические и эксплуатационные характеристики.

\begin{enumerate}
\item \textbf{По степени подобия (фидельности) реальному объекту}:
\begin{itemize}
\item \textit{Полномасштабные (высокой фидельности)}: Обеспечивают полное физическое, функциональное и визуальное подобие реального рабочего места (например, кабина самолета с полным набором штатного оборудования и системой подвижности). Требования к таким тренажерам наиболее жестко регламентируются отраслевыми стандартами (например, в авиации -- нормами лётной годности).
\item \textit{Парциальные (частичные)}: Моделируют отдельные системы, агрегаты или функции объекта (например, тренажер руления самолетом, тренажер работы с конкретной радиолокационной станцией). Позволяют отработать изолированные навыки.
\item \textit{Синтетические (компьютерные)}: Основной интерфейс представлен средствами компьютерной графики на мониторах, часто с использованием VR/AR технологий. Физическое подобие может быть ограничено. Обладают высокой гибкостью и относительно низкой стоимостью.
\end{itemize}
\item \textbf{По типу моделирования и обратной связи}:
\begin{itemize}
    \item \textit{Интерактивные (замкнутого цикла)}: Действия обучаемого напрямую влияют на поведение модели в реальном времени, система генерирует адекватную обратную связь по всем каналам (визуальному, звуковому, тактильному). Являются стандартом для комплексной подготовки.
    \item \textit{Демонстрационные (разомкнутого цикла)}: Предназначены в первую очередь для показа правильных действий и процедур. Взаимодействие обучаемого ограничено.
\end{itemize}

\item \textbf{По назначению и этапам подготовки} (в соответствии с \textbf{ГОСТ Р 55060-2012}):
\begin{itemize}
    \item \textit{Тренажеры для первичной подготовки}: Формирование базовых навыков управления, изучения интерфейсов и стандартных процедур.
    \item \textit{Тренажеры для поддержания и повышения квалификации}: Отработка уже сформированных навыков, включая действия в нештатных и аварийных ситуациях.
    \item \textit{Тренажеры для оценки и аттестации}: Используются для объективной проверки уровня подготовки оператора. Требуют высочайшей степени стандартизации и валидности моделирования.
\end{itemize}

\item \textbf{По уровню комплексности решаемых задач}:
\begin{itemize}
    \item \textit{Процедурные тренажеры}: Отработка последовательности действий по инструкциям и контрольным картам.
    \item \textit{Тактические и стратегические тренажеры}: Моделирование сложных сценариев, требующих принятия решений в условиях неполной информации, часто в многопользовательском режиме (например, тренажеры боевых операций, управления энергосистемой).
\end{itemize}
\end{enumerate}

\subsection{Архитектура тренажерных систем}

Архитектура современного тренажера, независимо от его типа, является модульной и, как правило, распределенной. Её можно представить в виде взаимодействия следующих основных подсистем, что также находит отражение в требованиях стандартов на тренажерные комплексы.

\begin{enumerate}
\item \textbf{Аппаратно-технический комплекс (АТК)}:
\begin{itemize}
\item \textit{Интерфейсное оборудование обучаемого}: Макеты кабин, пультов, органов управления (штурвалы, рычаги, кнопки, переключатели), снабженные реальными или эмитированными датчиками. Согласно эргономическим требованиям \textbf{ГОСТ Р ИСО 9241-400}, эти элементы должны обеспечивать соответствие антропометрическим данным оператора и требуемое усилие при воздействии.
\item \textit{Системы визуализации}: Проекционные системы, мониторы, дисплеи на лобовом стекле (HUD), шлемы виртуальной реальности. Должны обеспечивать заданное поле обзора, разрешение, яркость и частоту обновления, достаточные для создания эффекта присутствия и корректного восприятия информации.
\item \textit{Системы звукового и тактильного моделирования}: Акустические системы для воспроизведения шумов, голосов, сигналов; системы силовой обратной связи (хаптики) и вибрации. Их реализация критична для формирования полной сенсорной картины.
\item \textit{Система подвижности (опционально)}: Шестистепенные платформы, системы натяжения тросов и т.д., создающие физические ощущения ускорений, кренов, вибраций.
\end{itemize}
\item \textbf{Программно-математическое обеспечение (ПМО)}:
\begin{itemize}
    \item \textit{Модель объекта (Mathematical Model)}: Ядро системы. Детализированная математическая модель, описывающая физику, динамику, логику работы имитируемого объекта и его подсистем. В соответствии с \textbf{ГОСТ Р 55060-2012}, модель должна быть адекватной целям тренировки и обеспечивать необходимую точность имитации.
    \item \textit{Сценарный модуль (Instructor Station)}: Позволяет инструктору управлять ходом тренировки: задавать начальные условия, вводить неисправности, изменять параметры окружающей среды, ставить задачи обучаемому. Интерфейс инструктора должен быть интуитивно понятным и предоставлять полный контроль.
    \item \textit{Модуль регистрации, воспроизведения и анализа (Debriefing System)}: Осуществляет запись всех параметров тренировки, действий обучаемого и состояния модели для последующего детального разбора. Стандарты подчеркивают важность объективности и полноты протоколирования.
    \item \textit{Модуль оценки действий (Scoring Module)}: На основе заранее заданных критериев (время, точность, последовательность действий) автоматически оценивает performance обучаемого. Алгоритмы оценки должны быть прозрачными и валидными.
    \item \textit{Графический движок (Visual System)}: Генерирует изображение внешней и внутренней обстановки (кабины, местности, приборов) на основе данных модели объекта.
\end{itemize}

\item \textbf{Системные и коммуникационные средства}:
\begin{itemize}
    \item Обеспечивают синхронизацию и обмен данными между всеми модулями в реальном времени с минимальной задержкой (латентностью). Задержки в системе «действие -- реакция» строго лимитируются, так как их превышение разрушает иллюзию реальности и снижает качество обучения.
\end{itemize}
\end{enumerate}

Таким образом, архитектура тренажера представляет собой целостную систему, где человеко-машинный интерфейс является не просто одним из компонентов, а интегральной средой, связывающей аппаратуру, математические модели и когнитивные процессы обучаемого. Проектирование этой архитектуры требует соблюдения баланса между технической сложностью, педагогической эффективностью и экономической целесообразностью, что и фиксируется в комплексе соответствующих нормативных документов.

\newpage
\section{Педагогические аспекты и эффективность обучающих тренажерных систем}
Обучающая система на базе тренажера представляет собой синтез технического комплекса и дидактической методики. Её конечная эффективность определяется не только степенью технического подобия реальному объекту, но и тем, насколько грамотно реализованы педагогические принципы управления учебной деятельностью, формирования навыков и оценки результатов. Данные аспекты регламентируются как общими стандартами в области качества образования и эргономики, так и специализированными нормативами, такими как \textbf{ГОСТ Р 55060-2012 «Тренажерные комплексы. Общие требования к построению, функционированию и применению. Оценка соответствия»}.

\subsection{Педагогические принципы, реализуемые в тренажерах}

Эффективный тренажер должен быть построен на основе современных теорий обучения взрослых (андрагогики) и когнитивной психологии. Его архитектура должна воплощать следующие принципы:

\begin{enumerate}
\item \textbf{Принцип активности и вовлеченности (Learning by Doing)}: Обучаемый является не пассивным наблюдателем, а активным участником виртуальных событий. Его действия напрямую влияют на развитие сценария, что повышает мотивацию и степень усвоения материала. ГОСТ Р 55060-2012 подчеркивает необходимость обеспечения \textit{интерактивного режима} работы, при котором система реагирует на действия оператора в реальном времени.
\item \textbf{Принцип поэтапного формирования умственных действий (П.Я. Гальперин)}: Обучение на тренажере должно проходить от освоения отдельных операций к комплексной деятельности. Это реализуется через иерархию тренировочных сценариев:
\begin{itemize}
    \item \textit{Ознакомительные}: Изучение интерфейса, расположения органов управления, базовых процедур.
    \item \textit{Тренировочные}: Отработка стандартных (штатных) операций до автоматизма.
    \item \textit{Квалификационные и аттестационные}: Выполнение комплексных задач, включая действия в нештатных и аварийных ситуациях.
\end{itemize}

\item \textbf{Принцип адаптивности и индивидуализации}: Современные интеллектуальные обучающие системы (Intelligent Tutoring Systems, ITS) способны анализировать действия обучаемого в реальном времени и адаптировать сценарий: упрощать задачу при возникновении трудностей или, напротив, усложнять её для поддержания оптимального уровня когнитивной нагрузки (принцип «зоны ближайшего развития» Л.С. Выготского). ГОСТ Р 55060 косвенно поддерживает это требование, указывая на необходимость наличия у инструктора средств \textit{управления параметрами} тренировки.

\item \textbf{Принцип немедленной и содержательной обратной связи}: Одна из ключевых педагогических функций ЧМИ тренажера. Обратная связь должна быть:
\begin{itemize}
    \item \textit{Констатирующей}: Что произошло в результате действия.
    \item \textit{Корректирующей}: Была ли выполнена процедура верно, в чем ошибка.
    \item \textit{Проективной}: К каким последствиям приведет ошибка в реальных условиях.
\end{itemize}
Обратная связь предоставляется мультимодально: через изменение показаний приборов (визуальная), звуковые сигналы (аудиальная), тактильные ощущения (кинестетическая), а также через разбор полетов (debriefing).

\item \textbf{Принцип безопасности и безрисковости}: Фундаментальное преимущество тренажера. Обучаемый может совершать ошибки, ведущие к катастрофическим последствиям в реальности (срыв самолета, авария на АЭС), и на собственном опыте изучать их генезис и отработку правильных действий, не подвергая опасности себя и оборудование.
\end{enumerate}

\subsection{Методы оценки эффективности тренажерной подготовки}

Эффективность тренажера как обучающей системы нуждается в объективной оценке. Данный процесс должен быть системным и опираться на критерии, согласованные с целями подготовки. ГОСТ Р 55060-2012 вводит понятие \textbf{адекватности тренажера} – степени соответствия его характеристик поставленным целям обучения. Оценка эффективности проводится на нескольких уровнях:

\begin{enumerate}
\item \textbf{Оценка технической адекватности (Валидация модели)}: Проверка, насколько точно математическая модель и интерфейс тренажера воспроизводят характеристики реального объекта. Методы: сравнение лётных данных реального и виртуального самолета в идентичных режимах, экспертиза эргономики интерфейса на соответствие \textbf{ГОСТ Р ИСО 9241}.
\item \textbf{Оценка функциональной адекватности}: Определяет, позволяет ли тренажер отработать целевые навыки. Методы:
\begin{itemize}
    \item \textit{Экспертная оценка} инструкторов и опытных операторов.
    \item \textit{Метод контрольных групп}: Сравнение результатов подготовки группы, использовавшей тренажер, и группы, обучавшейся только традиционными методами.
\end{itemize}

\item \textbf{Оценка педагогической (дидактической) эффективности}: Измерение роста компетенций у конкретного обучаемого. Проводится с помощью встроенной \textbf{системы объективного оценивания (СОО)} тренажера, которая должна фиксировать:
\begin{itemize}
    \item Временные параметры (задержка реакции, время выполнения процедуры).
    \item Точность действий (отклонение от заданных параметров, количество и тип ошибок).
    \item Степень соответствия действий регламентированным алгоритмам (алгоритмичность).
    \item Психофизиологические показатели (в сложных тренажерах) – частота сердечных сокращений, кожно-гальваническая реакция, траектория взгляда (айтрекинг).
\end{itemize}
На основе этих данных формируется \textit{интегральный показатель успешности (ИПУ)} и детализированный протокол разбора.

\item \textbf{Последейственная оценка (Transfer of Training)}: Наиболее значимый критерий. Оценивает, насколько навыки, сформированные на тренажере, переносятся на реальную технику. Определяется по статистике успешности действий выпускников в реальных условиях или на эталонных тренажерах высшего класса.
\end{enumerate}

\subsection{Интеграция тренажера в единую систему подготовки}

Тренажер не является изолированным инструментом. Его максимальная эффективность достигается при интеграции в \textbf{единую информационно-обучающую среду (E-learning Environment)}. В такую среду, помимо самого тренажера, могут входить:
\begin{itemize}
\item Электронные учебные курсы (Теория) с интерактивными модулями.
\item Системы управления обучением (LMS), планирующие индивидуальные траектории.
\item Базы данных сценариев и результатов тренировок.
\item Средства вебинаров и дистанционного контроля для инструктора.
\end{itemize}

Таким образом, современная обучающая тренажерная система представляет собой сложный человеко-машинный комплекс, где техническое исполнение интерфейса подчинено фундаментальным педагогическим задачам. Её проектирование требует междисциплинарного подхода, объединяющего инженеров, математиков-моделистов, эргономистов и специалистов в области педагогической психологии. Соблюдение стандартов, таких как ГОСТ Р 55060, обеспечивает системность этого процесса и гарантирует, что созданный тренажер будет не просто симулятором, а эффективным инструментом формирования профессионального мастерства.
\section{Классификация обучающих систем в контексте человеко-машинного интерфейса}

Обучающие системы (ОС) в сфере человеко-машинного интерфейса представляют собой класс программных и аппаратно-программных комплексов, предназначенных для формирования, закрепления и оценки знаний, умений и навыков (компетенций) оператора, связанных с взаимодействием с конкретной технической системой или классом систем. Тренажеры (симуляторы) являются важным, но не единственным подвидом таких систем. Классификация обучающих систем может быть проведена по ряду ключевых признаков, отражающих их педагогическую философию, архитектурные решения и роль в учебном процессе. Данный подход согласуется с общими принципами \textbf{ГОСТ Р ИСО 9241-210-2019}, который рассматривает проектирование интерфейсов как процесс, ориентированный на цели и задачи пользователя, где обучение является одной из ключевых задач.

\subsection{Классификация по степени автоматизации и интеллектуализации}

Данный критерий определяет уровень самостоятельности системы в управлении учебным процессом.

\begin{enumerate}
\item \textbf{Пассивные (информационно-справочные) системы}:
\begin{itemize}
\item \textbf{Описание}: Предоставляют структурированный учебный материал (текст, видео, интерактивные схемы) без адаптации под действия обучаемого. Обратная связь отсутствует или минимальна (например, тесты с автоматической проверкой).
\item \textbf{Примеры}: Электронные учебники, интерактивные технические руководства (IETM - Interactive Electronic Technical Manual), библиотеки видеолекций по устройству системы.
\item \textbf{Роль ЧМИ}: Интерфейс ориентирован на эффективную навигацию и восприятие информации. Ключевые требования — ясность, структурированность, соответствие принципам \textbf{ГОСТ Р ИСО 9241-112} (принципы представления информации).
\end{itemize}
\item \textbf{Интерактивные практикумы}:
\begin{itemize}
    \item \textbf{Описание}: Позволяют обучаемому выполнять действия в смоделированной или упрощенной среде (часто пошагово). Система предоставляет контекстно-зависимую помощь и проверяет правильность последовательности.
    \item \textbf{Примеры}: Виртуальные лабораторные работы, интерактивные процедурные тренажеры для отработки последовательности включения оборудования.
    \item \textbf{Роль ЧМИ}: Интерфейс должен обеспечивать прямую манипуляцию элементами системы, немедленную и понятную обратную связь. Актуальны принципы \textbf{ошибкоустойчивости} и \textbf{обучаемости} из \textbf{ГОСТ Р ИСО 9241-110}.
\end{itemize}

\item \textbf{Интеллектуальные обучающие системы (Intelligent Tutoring Systems - ITS)}:
\begin{itemize}
    \item \textbf{Описание}: Наиболее сложный класс. Включают модель предметной области, модель обучаемого (учет знаний, ошибок, стиля обучения) и педагогическую модель (стратегия обучения). Адаптируют содержание, сложность и темп обучения в реальном времени.
    \item \textbf{Примеры}: Системы для подготовки операторов сложных пультов управления, способные выявлять пробелы в понимании и предлагать целевые упражнения.
    \item \textbf{Роль ЧМИ}: Диалоговый интерфейс, часто с элементами естественного языка или когнитивного помощника. Ключевое требование — \textbf{индивидуализируемость} и \textbf{соответствие ожиданиям} (\textbf{ГОСТ Р ИСО 9241-110}).
\end{itemize}

\end{enumerate}

\subsection{Классификация по глубине погружения и типу среды}

Критерий отражает способ создания учебного контекста и вовлеченности оператора.

\begin{enumerate}
\item \textbf{Системы на базе дополненной реальности (AR)}:
\begin{itemize}
\item \textbf{Описание}: Накладывают учебную информацию (подсказки, схемы, анимации) на изображение реального оборудования или рабочего места.
\item \textbf{Педагогическое назначение}: Обучение в контексте реального оборудования, поддержка при выполнении сложных процедур (например, сборка, ремонт).
\item \textbf{Требования к ЧМИ}: Минимальная интерференция с основным рабочим интерфейсом, точность привязки к объектам, эргономичность носимых устройств.
\end{itemize}
\item \textbf{Системы на базе виртуальной реальности (VR)}:
\begin{itemize}
    \item \textbf{Описание}: Полное погружение в синтетическую среду, имитирующую реальную или абстрактную систему.
    \item \textbf{Педагогическое назначение}: Отработка действий в условиях, трудновоспроизводимых в реальности (опасные ситуации, масштабные объекты), формирование пространственных навыков.
    \item \textbf{Требования к ЧМИ}: Устранение киберболезни, реалистичность тактильной и силовой обратной связи (хаптики), интуитивность взаимодействия со средой.
\end{itemize}

\item \textbf{Смешанные (гибридные) системы}:
\begin{itemize}
    \item \textbf{Описание}: Комбинируют физические макеты (частичные тренажеры) с виртуальными элементами, проецируемыми интерфейсами и т.д.
    \item \textbf{Педагогическое назначение}: Сочетание тактильного опыта работы с реальными элементами управления и гибкости виртуальной визуализации.
    \item \textbf{Пример}: Тренажер с физической панелью переключателей и виртуальными (проецируемыми) показаниями приборов.
\end{itemize}
\end{enumerate}

\subsection{Классификация по архитектуре и способу интеграции}

Данный критерий определяет, как система встроена в общий процесс подготовки.

\begin{enumerate}
\item \textbf{Автономные (изолированные) системы}:
\begin{itemize}
\item Решают конкретную учебную задачу (например, изучение одной подсистемы). Не обмениваются данными с другими системами.
\end{itemize}
\item \textbf{Интегрированные в единую обучающую среду (LMS - Learning Management System)}:
\begin{itemize}
    \item Являются модулем в составе платформы, которая управляет расписанием, хранит результаты, формирует индивидуальные траектории. Обмен данными (например, результаты тренировки на тренажере автоматически заносятся в журнал LMS) критически важен. Стандарты типа \textbf{SCORM (Sharable Content Object Reference Model)} или \textbf{xAPI (Experience API)} регламентируют этот обмен.
\end{itemize}

\item \textbf{Распределенные многопользовательские системы}:
\begin{itemize}
    \item Предназначены для отработки коллективных действий (экипажей, команд управления). Моделируют не только объект, но и взаимодействие между операторами. Требования к ЧМИ включают средства коммуникации и согласования действий в рамках интерфейса.
\end{itemize}
\end{enumerate}

Таким образом, классификация обучающих систем выявляет их разнообразие, простирающееся от простых электронных пособий до сложных интеллектуальных и иммерсивных комплексов. Выбор конкретного типа системы определяется триадой «задача–пользователь–контекст». Независимо от типа, проектирование их человеко-машинного интерфейса должно подчиняться общим эргономическим принципам \textbf{ГОСТ Р ИСО 9241}, а также учитывать специфические педагогические требования, направленные на обеспечение эффективного переноса сформированных компетенций в реальную профессиональную деятельность.

\newpage
\section{Анализ реальной обучающей системы: Полномасштабный авиационный тренажер Boeing 737 NG (FFS Level D)}

В качестве эталонного примера современной высокофидельной обучающей системы рассмотрим полномасштабный тренажер самолета Boeing 737 Next Generation (NG), сертифицированный по высшему уровню D в соответствии со стандартами Управления гражданской авиации США (FAA) и Европейского агентства авиационной безопасности (EASA). Данный тип тренажера (Full Flight Simulator, FFS) представляет собой максимально сложную и технически совершенную реализацию принципов человеко-машинного интерфейса в обучающих целях и является обязательным элементом подготовки и регулярной аттестации коммерческих пилотов во всем мире.

\subsection{Назначение и место в системе подготовки}

Тренажер уровня D предназначен для выполнения всей программы летной подготовки, за исключением первых этапов ознакомления с аэродинамикой. Его ключевые задачи:
\begin{itemize}
\item Первоначальное и переходное обучение пилотов на тип воздушного судна (типовая подготовка).
\item Регулярные тренировки (раз в 6 месяцев) для поддержания и повышения квалификации, включая отработку действий в нештатных и аварийных ситуациях.
\item Оценка и проверка (чек-райд) экипажей перед назначением на рейсы или после перерыва в летной работе.
\item Отработка специфических процедур для конкретного авиапредприятия (стандарты эксплуатации).
\end{itemize}
Сертификация уровня D означает, что все учебные часы, налетанные на таком тренажере, полностью засчитываются регулятором в качестве реального налета, что подчеркивает его абсолютную адекватность для формирования компетенций.

\subsection{Техническая архитектура системы}

Архитектура тренажера является модульной и строится вокруг принципа максимального физического и функционального подобия.

\subsubsection{Аппаратно-технический комплекс}

\begin{enumerate}
\item \textbf{Кабина (Cockpit Module)}:
\begin{itemize}
\item Точная реплика кабины Boeing 737 NG, включающая все органы управления: штурвальную колонку (или сайдстик), педали руля направления, рычаги управления двигателями (РУДы), более 600 переключателей, кнопок и тумблеров.
\item Используется реальное авиационное оборудование, снятое с производства или изготовленное по лицензии: дисплеи EFIS (Electronic Flight Instrument System), блоки управления FMS (Flight Management System), радиооборудование.
\item Все элементы обладают корректными тактильными характеристиками (усилие, ход, щелчки).
\end{itemize}
\item \textbf{Система визуализации (Visual System)}:
\begin{itemize}
    \item \textbf{Программная часть}: Используется высокодетализированная база данных аэропортов и местности (до уровня отдельных огней на ВПП и зданий), атмосферные эффекты (дождь, снег, туман, гроза) с корректным влиянием на видимость.
    \item \textbf{Аппаратная часть}: Как правило, система проекторов с цилиндрическим или сферическим экраном, обеспечивающая горизонтальное поле обзора не менее 200 градусов и вертикальное – не менее 40 градусов. Современные системы используют технологию коллимированных дисплеев, где изображение формируется «на бесконечности», что устраняет параллакс и повышает реализм.
    \item Часто реализована система ночного видения и моделирование особых визуальных эффектов (например, блики от солнца, огни города).
\end{itemize}

\item \textbf{Система движения (Motion System)}:
\begin{itemize}
    \item Шестистепенная гидравлическая или электрическая платформа (платформа Стюарта), способная воспроизводить линейные ускорения и угловые перемещения.
    \item Используется алгоритм «подсказки движением» (motion cueing): платформа не пытается имитировать длительные перегрузки (это физически невозможно), а дает начальный толчок, создающий у пилота физиологическое ощущение ускорения, а затем плавно возвращается в нейтральное положение, готовясь к следующему «сигналу».
\end{itemize}

\item \textbf{Звуковая система (Aural Cueing System)}:
\begin{itemize}
    \item Трехмерная система объемного звука, точно воспроизводящая шумы работы двигателей, аэродинамический шум, звуки выпуска шасси и закрылков, голосовое оповещение системы EGPWS («Pull up!»), сигналы радиообмена и звуки в кабине (например, щелчки переключателей).
\end{itemize}

\end{enumerate}

\subsubsection{Программно-математическое обеспечение}

\begin{enumerate}
\item \textbf{Модель полета (Flight Dynamics Model - FDM)}:
\begin{itemize}
\item Высокоточная нелинейная математическая модель, учитывающая аэродинамику, тягу двигателей, массо-инерционные характеристики, влияние конфигурации (выпуск закрылков, шасси) и атмосферных условий.
\item Модель проходит валидацию по данным реальных испытаний самолета и должна обеспечивать соответствие характеристик в пределах строгих допусков, установленных стандартами.
\end{itemize}
\item \textbf{Модели систем самолета (Systems Simulation)}:
\begin{itemize}
    \item Детально моделируются все бортовые системы: гидравлика, электрика, топливная система, кондиционирование, противообледенительная система и т.д. Это позволяет реалистично имитировать любые отказы.
\end{itemize}

\item \textbf{Рабочее место инструктора (Instructor Operating Station - IOS)}:
\begin{itemize}
    \item Представляет собой комплекс из нескольких мониторов и средств управления, позволяющий инструктору:
    \begin{itemize}
        \item Устанавливать начальные условия (аэродром, погоду, массу, балансировку).
        \item Вводить неисправности (более 500 возможных) в любой момент времени.
        \item Управлять действиями других виртуальных воздушных судов и служб управления воздушным движением (УВД).
        \item В реальном времени наблюдать за параметрами полета и действиями экипажа.
        \item Ставить «заморозку» (pause) для разбора ситуации и делать пометки.
    \end{itemize}
\end{itemize}

\item \textbf{Система записи и воспроизведения (Debriefing System)}:
\begin{itemize}
    \item Осуществляет синхронную запись всех параметров полета (сотни каналов), речевых переговоров в кабине и с УВД, а также видео с виртуальной точки зрения. После тренировки экипаж и инструктор могут детально разобрать каждый этап, используя графики, траекторию полета на карте и воспроизведение.
\end{itemize}
\end{enumerate}

\subsection{Педагогические аспекты и построение учебного процесса}

Обучение строится по принципу «от простого к сложному» и строго регламентировано учебными планами (Training Syllabi).

\begin{enumerate}
\item \textbf{Структура занятия}: Стандартная сессия длится 4 часа, из которых примерно 2,5-3 часа — непосредственно «полет», а остальное время — предварительный брифинг и последующий детальный разбор (дебрифинг).
\item \textbf{Роль инструктора}: Инструктор (Trainer) выступает не только как контролер, но и как фасилитатор. Он управляет сложностью сценария, дает подсказки, создает психологическую нагрузку, имитируя стрессовые условия (например, имитация срочности при отказе).
\item \textbf{Типовые сценарии}:
\begin{itemize}
\item \textbf{Нормальные процедуры}: Взлет, набор высоты, крейсерский полет, снижение, заход на посадку, посадка в различных погодных условиях.
\item \textbf{Нештатные ситуации}: Отказ одного двигателя на взлете (V1 cut), отказ гидросистемы, пожары, разгерметизация, ложные показания приборов.
\item \textbf{Аварийные процедуры}: Посадка на неисправном шасси, вынужденная посадка на воду, действия при столкновении с птицами.
\end{itemize}
\end{enumerate}

\subsection{Соответствие стандартам и нормам}

Тренажер сертифицируется по строгим авиационным правилам, которые являются аналогом ГОСТ в этой области.

\begin{itemize}
\item \textbf{FAA Advisory Circular 120-40B / EASA CS-FSTD(A)}: Детально регламентируют требования к тренажерам уровня D, включая:
\begin{itemize}
\item Точность модели полета (допустимые отклонения от данных реального самолета).
\item Минимальное поле обзора визуальной системы, разрешение, задержку (латентность < 150 мс).
\item Характеристики системы движения (диапазоны перемещений, ускорений).
\item Требования к моделированию систем и окружающей среды.
\end{itemize}
\item В России аналогичные требования содержатся в \textbf{Авиационных правилах АП-23 (Нормы летной годности)} и \textbf{АП-ТРЕН (Требования к тренажерам)}. Тренажер должен пройти процедуру квалификации и регулярной периодической проверки.
\end{itemize}

\subsection{Пример конкретного учебного сценария: Отработка отказа двигателя на взлете}

\begin{enumerate}
\item \textbf{Предварительный брифинг}: Инструктор ставит задачу: выполнить нормальный взлет с аэродрома XYZ. Пилоты изучают данные о массе, балансировке, погоде, рассчитывают скорости V1 (скорость принятия решения), VR (скорость отрыва).
\item \textbf{Выполнение}:
\begin{itemize}
\item Пилоты выполняют руление, занимают исполнительный старт. Командир экипажа (КВС) выполняет взлет.
\item В момент достижения скорости V1 (заранее запрограммированный момент) инструктор с IOS вводит отказ одного двигателя (например, пожар левого двигателя). Система воспроизводит звуковую и визуальную сигнализацию (сирена пожара, мигающая кнопка), возникает разворачивающий момент и потеря тяги.
\item КВС должен немедленно подтвердить отказ («Engine fire left!»), продолжить взлет, так как скорость уже превышает V1 (прерывать взлет опасно), убрать тягу отказавшего двигателя, активировать систему пожаротушения, выполнить набор высоты с одним двигателем по заданной процедуре.
\item Второй пилот (помощник) выполняет контрольный список (checklist) по устранению пожара и конфигурации самолета для полета на одном двигателе.
\end{itemize}
\item \textbf{Дебрифинг}:
\begin{itemize}
\item После посадки система воспроизводит запись. Инструктор совместно с экипажем анализирует:
\begin{itemize}
\item Время реакции на отказ.
\item Правильность и последовательность действий.
\item Точность выдерживания параметров полета (скорость, крен, траектория).
\item Эффективность коммуникации в кабине (Crew Resource Management).
\end{itemize}
\item Строятся графики отклонений от заданной траектории, что дает объективную основу для оценки.
\end{itemize}
\end{enumerate}

Таким образом, полномасштабный тренажер Boeing 737 NG Level D представляет собой вершину эволюции обучающих систем, где человеко-машинный интерфейс достиг такого уровня совершенства, что граница между тренировкой и реальностью в контексте формирования профессиональных навыков становится практически неразличимой. Это делает его незаменимым инструментом обеспечения безопасности и эффективности современной гражданской авиации.

\newpage
\section{Анализ реальной обучающей системы: Роботизированный хирургический симулятор da Vinci Skills Simulator}

В качестве второго примера высокотехнологичной обучающей системы рассмотрим симулятор для отработки навыков роботизированной хирургии da Vinci Skills Simulator. Данная система является специализированным тренажером, интегрируемым в консоль хирургической роботизированной системы da Vinci, и предназначена для подготовки и сертификации хирургов в области минимально инвазивных операций. Она представляет собой яркий пример применения принципов ЧМИ в высокорисковой профессиональной области, где цена ошибки чрезвычайно высока, а обучение на пациентах недопустимо.

\subsection{Назначение и целевая аудитория}

Система предназначена для:
\begin{itemize}
\item Первичного обучения хирургов работе с интерфейсом и органами управления роботизированной системой da Vinci.
\item Отработки базовых и продвинутых моторных навыков, необходимых для манипуляций в трехмерном пространстве с ограниченной тактильной обратной связью.
\item Подготовки к выполнению конкретных хирургических процедур (например, холецистэктомии, простатэктомии).
\item Объективной оценки уровня мастерства и допуска к реальным операциям. Результаты тренировок часто используются в рамках формализованных программ сертификации.
\end{itemize}
Целевая аудитория – хирурги-ординаторы, а также опытные хирурги, осваивающие новую технологию.

\subsection{Архитектура и компоненты системы}

Система реализует гибридный подход, совмещая реальный аппаратный интерфейс и виртуальную симуляцию.

\subsubsection{Аппаратная часть}

\begin{enumerate}
\item \textbf{Консоль хирурга (Surgeon Console)}: Используется штатная консоль системы da Vinci. Обучаемый располагается у консоли, помещает лицо в стереоскопический визуализатор, а руки – в мастер-манипуляторы. Это обеспечивает полное физическое соответствие реальному рабочему месту. Все органы управления (педали для коагуляции и камеры, переключатели) функциональны.
\item \textbf{Модуль симулятора (Simulator Unit)}: Специальный модуль, устанавливаемый на место реальных инструментов и камеры пациента. Содержит сенсоры, отслеживающие движения мастер-манипуляторов, и может обеспечивать элементарную силовую обратную связь для имитации контакта с виртуальными тканями.
\end{enumerate}

\subsubsection{Программное обеспечение и учебный контент}

\begin{enumerate}
\item \textbf{Библиотека упражнений}: Включает десятки виртуальных заданий, разбитых на категории по сложности:
\begin{itemize}
\item \textbf{Базовые навыки}: Управление камерой, координация движений, захват и перемещение предметов, наложение швов на виртуальные ткани.
\item \textbf{Продвинутые навыки}: Рассечение с точным контролем глубины, электрокоагуляция, работа в условиях кровотечения.
\item \textbf{Процедурные симуляции}: Виртуальные реплики этапов реальных операций.
\end{itemize}
\item \textbf{Система объективной оценки (Metrics Engine)}: Ядро обучающей системы. В режиме реального времени отслеживает и анализирует сотни параметров:
\begin{itemize}
    \item \textit{Экономия движений}: Длина пути инструментов, количество избыточных движений.
    \item \textit{Точность}: Отклонения от оптимальной траектории, точность попадания в целевую зону.
    \item \textit{Эффективность}: Время выполнения задачи, процент выполнения этапов.
    \item \textit{Безопасность}: Сила сжатия инструментов (имитация), контакт с запрещенными зонами, повреждение окружающих виртуальных тканей.
\end{itemize}

\item \textbf{Интерфейс инструктора и система отчетности}: Позволяет преподавателю выбирать упражнения, устанавливать параметры (например, «сложность кровотечения»). После выполнения задания формируется детализированный отчет с графиками и численными показателями, который используется для структурированного разбора.
\end{enumerate}

\subsection{Педагогическая методология}

Обучение построено на принципах deliberate practice (целенаправленной практики) и мастерства.

\begin{enumerate}
\item \textbf{Поэтапное освоение}: Хирург не может перейти к сложным процедурам, не отработав базовые движения до автоматизма и не достигнув пороговых значений по ключевым метрикам.
\item \textbf{Немедленная и количественная обратная связь}: После каждого упражнения система выдает оценку (часто в виде баллов или «звезд») и показывает, по каким конкретно параметрам были допущены отклонения. Это позволяет фокусировать усилия на слабых местах.
\item \textbf{Безопасная среда для ошибок}: Хирург может многократно «разрывать» виртуальный сосуд или «прокалывать» ткань, анализируя последствия и отрабатывая корректирующие действия, что абсолютно невозможно в операционной.
\item \textbf{Формирование внутренней ментальной модели}: Система способствует формированию правильной связи между визуальным стереоизображением, кинестетикой движений манипуляторов и моторным ответом, что критически важно для роботизированной хирургии.
\end{enumerate}

\subsection{Соответствие профессиональным стандартам}

Система de facto стала отраслевым стандартом подготовки. Её использование регламентировано внутренними стандартами многих клиник и хирургических ассоциаций. Разрабатываемые на её основе программы часто соответствуют требованиям к непрерывному медицинскому образованию (CME). Эффективность обучения подтверждается клиническими исследованиями, демонстрирующими положительную корреляцию между результатами на симуляторе и качеством выполнения реальных операций.

Таким образом, da Vinci Skills Simulator представляет собой не просто «тренажер», а комплексную адаптивную среду для формирования и объективной оценки высокоуровневых психомоторных навыков. Она наглядно демонстрирует, как глубоко интегрированный человеко-машинный интерфейс, снабженный системой аналитики, трансформирует процесс профессионального обучения, делая его измеримым, безопасным и исключительно эффективным.

\newpage
\section*{Заключение}
\addcontentsline{toc}{section}{Заключение}
Проведенный анализ позволяет утверждать, что обучающие системы, основанные на технологиях симуляции и глубокой интеграции человеко-машинного интерфейса, представляют собой закономерное и мощное развитие принципов ЧМИ, ориентированных на специфическую задачу – формирование профессионального мастерства. Из простого инструмента передачи информации они эволюционировали в сложные киберфизические комплексы, где аппаратное воплощение, математическое моделирование и педагогические методики сливаются в единую интерактивную среду.

Ключевым выводом является то, что эффективность такой системы определяется не максимальной технической сложностью, а ее адекватностью поставленным учебным целям. Классификация систем по степени фидельности, интеллектуализации и типу среды (AR/VR) предоставляет методический аппарат для этого выбора. Высокофидельные тренажеры, подобные авиационному FFS Level D, незаменимы для комплексной отработки сенсомоторных навыков и действий в стрессовых условиях, где требуется полное погружение и реализм. В то же время, системно-ориентированные интеллектуальные симуляторы, такие как da Vinci Skills Simulator, демонстрируют, что даже при относительной простоте визуализации, глубокая обратная связь и объективная метрическая оценка каждого действия могут обеспечить исключительную эффективность формирования специфических моторных и когнитивных навыков.

Архитектура современных обучающих систем, как было показано, строится на триаде: реалистичный интерфейс взаимодействия, адекватная математическая модель предметной области и подсистема педагогического управления и анализа. Именно последний компонент превращает симулятор из пассивного имитатора в активного «интеллектуального инструктора», способного адаптировать сценарий, выявлять слабые места обучаемого и предоставлять структурированные данные для разбора.

Важнейшим аспектом, подтвержденным анализом, является необходимость строгого нормативного регулирования, будь то авиационные правила (FAA/EASA), медицинские стандарты или общие эргономические принципы ГОСТ Р ИСО 9241. Эти стандарты обеспечивают не только безопасность и надежность, но и, что критически важно, валидность обучения – доказательство того, что навыки, отработанные на тренажере, действительно переносятся на реальную деятельность.

Таким образом, обучающие системы на базе тренажеров воплощают в себе высшую форму человеко-машинного интерфейса, где интерфейс перестает быть границей между человеком и машиной, а становится прозрачной средой для целенаправленного формирования опыта. Их развитие движется в сторону большей адаптивности, интеллектуализации и интеграции в единые цифровые образовательные экосистемы. Дальнейшее совершенствование этих систем, основанное на междисциплинарном синтезе эргономики, педагогики, компьютерного моделирования и data science, является ключевым фактором повышения уровня профессионализма, безопасности и эффективности в наиболее ответственных областях человеческой деятельности.

\newpage
\addcontentsline{toc}{section}{Список литературы}
\begin{thebibliography}{9}
    \bibitem{cooper}
    Алан Купер. \emph{Об интерфейсе. Основы проектирования взаимодействия}. - 2-е изд. - М.: ИМВ, 2009. — 688 с.
    \end{thebibliography}

\end{document}