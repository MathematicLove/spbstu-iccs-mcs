\documentclass[areasetadvanced]{scrartcl}

\usepackage[utf8]{inputenc}
\usepackage[T2A]{fontenc}
\usepackage[english,russian]{babel}

\usepackage[footskip=1cm,left=25mm, right=15mm, top=20mm, bottom=20mm]{geometry}
\usepackage{setspace}
\usepackage{amsmath, amssymb} 
\usepackage{graphicx}
\usepackage{tikz}
\usetikzlibrary{arrows.meta}
\usepackage{float}
\usepackage{dashrule}
\usepackage{fancyhdr} 
\usepackage{hyperref} 
\usepackage{parskip}
\usepackage{textcomp, enumitem}
\usepackage{indentfirst}
\usepackage{graphicx}
\usepackage{algorithm}
\usepackage{algpseudocode}
\usepackage{array} 
\usepackage{geometry}
\usepackage{afterpage}
\usepackage{minted}
\setcounter{secnumdepth}{3} 
\setcounter{tocdepth}{3}    
\usepackage{listings} 

\newcommand{\icon}[1]{\includegraphics[height=1.2em]{\#1}}

\tikzstyle{block} = [rectangle, rounded corners, minimum width=3cm, minimum height=1cm, text centered, draw=black, fill=lightgray]

\setkomafont{sectioning}{\normalfont\bfseries} 
\setkomafont{section}{\normalfont\Large\bfseries}
\setkomafont{subsection}{\normalfont\large\bfseries}
\setkomafont{subsubsection}{\normalfont\large\bfseries}
\setkomafont{paragraph}{\normalfont\large\bfseries} 

\lstset{
  language=Haskell,
  basicstyle=\ttfamily\small,
  keywordstyle=\color{blue}\bfseries,
  stringstyle=\color{red},
  commentstyle=\color{green!70!black},
  numbers=left,
  numberstyle=\tiny,
  stepnumber=1,
  numbersep=10pt,
  showstringspaces=false,
  breaklines=true,
  frame=single
}

\lstdefinelanguage{Lua}{
    keywords={function, end, if, then, else, elseif, for, while, do, repeat, until, break, return, local, and, or, not, true, false, nil},
    keywordstyle=\color{blue}\bfseries,
    stringstyle=\color{red},
    commentstyle=\color{green!70!black},
    morestring=[s]{"}{"},
    morestring=[s]{'}{'},
    morecomment=[l]{--},
    morecomment=[s]{--[[}{]]},
    basicstyle=\ttfamily\small,
    numbers=left,
    numberstyle=\tiny,
    stepnumber=1,
    numbersep=10pt,
    showstringspaces=false,
    breaklines=true,
    frame=single
}

\setlength{\parindent}{1.25cm}
\setcounter{tocdepth}{2}
\begin{document}
\sloppy
	\thispagestyle{empty}
	\begin{center}
		\large{МИНОБРНАУКИ РОССИИ} \par
		\vspace{0.3cm}
		\normalsize
		{ФЕДЕРАЛЬНОЕ ГОСУДАРСТВЕННОЕ АВТОНОМНОЕ ОБРАЗОВАТЕЛЬНОЕ УЧРЕЖДЕНИЕ ВЫСШЕГО ОБРАЗОВАНИЯ} \par
		\vspace{0.3cm}
		\textbf{\guillemotleft САНКТ-ПЕТЕРБУРГСКИЙ ПОЛИТЕХНИЧЕСКИЙ}
		\textbf{УНИВЕРСИТЕТ ПЕТРА ВЕЛИКОГО\guillemotright} \par
		\vspace{0.3cm}
		{Институт компьютерных наук и кибербезопасности}\par
		{Высшая школа технологий искусственного интеллекта}\par
	\end{center}
	\vfill
	\begin{center}

        \par
		{\Huge ДОКЛАД}\par
        \large {\guillemotleft Сообщение об ошибках, уведомление, подтверждение\guillemotright}\par
		\Large {Человеко-машинный интерфейс}
	\end{center}
	\vfill
	\begin{flushleft}
		Студент: \hspace{1.8cm} \rule[0pt]{2.5cm}{0.5pt}\hfill Салимли Айзек Мухтар Оглы\par
		\vspace{1.5cm}
		Преподаватель: \hspace{0.55cm} \rule[0pt]{2.5cm}{0.5pt}\hfill  Курочкин Михаил Александрович
	\end{flushleft}
	\vspace{0.5cm}
	\begin{flushright}
		\guillemotleft \rule[0pt]{0.8cm}{0.5pt}\guillemotright \rule[0pt]{2cm}{0.5pt} 20\rule[0pt]{0.5cm}{0.5pt} г.
	\end{flushright}
	\vfill
	\begin{center}
		Санкт-Петербург, 2025
	\end{center}
	\newpage
	\tableofcontents
\newpage
\section*{Введение}
\addcontentsline{toc}{section}{Введение}
	
Человеко-машинный интерфейс (ЧМИ) --- это методы и средства обеспечения непосредственного взаимодействия между оператором и технической системой, представляющих возможности оператору управлять этой системой и контролировать ее работу. Качество ЧМИ напрямую влияет на эффективность освоения программного обеспечения и продуктивность работы пользователя. В контексте проектирования взаимодействия, одной из ключевых задач ЧМИ является обеспечение непрерывной, понятной и непротиворечивой обратной связи, которая позволяет пользователю корректно оценивать состояние системы и результаты своих действий.

Особую роль в этой обратной связи играют три взаимосвязанных класса элементов: \textbf{уведомления}, \textbf{подтверждения} и \textbf{сообщения об ошибках}. Эти компоненты формируют критически важный диалог между пользователем и системой, направленный на поддержание контроля, предотвращение ошибочных действий и минимизацию когнитивной нагрузки. Уведомления информируют о событиях и изменениях состояния, подтверждения требуют санкции на выполнение потенциально необратимых или значимых операций, а сообщения об ошибках сигнализируют о проблемах, препятствующих выполнению задачи, и предлагают пути их решения.

Эффективная реализация данных элементов требует учета не только технических аспектов, но и психофизиологических особенностей человека, таких как восприятие, внимание и память. Грамотно спроектированные, они повышают надежность, безопасность и удобство системы; плохо реализованные --- становятся источником разочарования, стресса и новых ошибок. Таким образом, анализ принципов, паттернов и лучших практик проектирования ошибок, уведомлений и подтверждений представляет собой существенный аспект теории и практики разработки качественных человеко-машинных интерфейсов.

\newpage
\section{Стандарт ГОСТ Р ИСО 9241 в контексте проектирования элементов обратной связи}

Стандарт ГОСТ Р ИСО 9241 «Эргономика взаимодействия человек-система» представляет собой адаптированный комплекс национальных российских стандартов, основанный на международной серии ISO 9241. Данный стандарт является фундаментальным документом, устанавливающим эргономические требования и рекомендации по проектированию человеко-машинных интерфейсов. Его применение обеспечивает системный, научно обоснованный подход к созданию интерфейсов, ориентированных на пользователя. Для проектирования таких элементов обратной связи, как ошибки, уведомления и подтверждения, наибольшее значение имеют следующие части стандарта:

\subsection{ГОСТ Р ИСО 9241-110: Принципы организации диалога}

Данная часть стандарта устанавливает семь основополагающих принципов диалога, которые напрямую определяют качество взаимодействия с элементами обратной связи.

\begin{itemize}
\item \textbf{Пригодность для выполнения задачи (Suitability for the task)}. Диалог должен поддерживать пользователя в эффективном выполнении задачи. Это означает, что:
\begin{itemize}
\item Уведомления появляются тогда, когда информация критична для продолжения задачи.
\item Подтверждения запрашиваются только для потенциально критических или необратимых действий, а не для каждого шага.
\item Сообщения об ошибках помогают решить проблему и продолжить выполнение задачи.
\end{itemize}
\item \textbf{Самодокументируемость (Self-descriptiveness)}. Каждый шаг диалога должен быть понятным посредством предоставления соответствующей обратной связи или пояснений по запросу.
\begin{itemize}
    \item Сообщения об ошибках должны быть самодостаточными, объясняя причину и путь решения.
    \item Уведомления должны четко указывать на вызвавшее их событие.
    \item В диалогах подтверждения однозначно формулируется суть запрашиваемого действия.
\end{itemize}

\item \textbf{Управляемость (Controllability)}. Пользователь должен иметь возможность управлять темпом и последовательностью взаимодействия.
\begin{itemize}
    \item Предоставление возможности отменить действие после подтверждения (где это применимо).
    \item Немодальные уведомления позволяют пользователю самому решить, когда на них отреагировать.
    \item Наличие безопасного выхода («Отмена») из любого диалога подтверждения или ошибки.
\end{itemize}

\item \textbf{Соответствие ожиданиям пользователя (Conformity with user expectations)}. Диалог должен соответствовать общепринятым соглашениям и опыту пользователя.
\begin{itemize}
    \item Использование устоявшихся иконок и цветов (красный для ошибок, желтый для предупреждений, зеленый для успеха).
    \item Следование последовательности «инициатива -- действие -- обратная связь».
    \item Размещение кнопок подтверждения в соответствии с платформенными соглашениями (например, «ОК/Принять» слева или справа).
\end{itemize}

\item \textbf{Ошибкоустойчивость (Error tolerance)}. Диалог должен минимизировать возникновение ошибок и помогать в их исправлении.
\begin{itemize}
    \item Предотвращение ошибок через подтверждения.
    \item Конструктивные сообщения об ошибках, способствующие их исправлению.
    \item Возможность легкого восстановления системы после ошибки.
\end{itemize}

\item \textbf{Индивидуализируемость (Individualization)}. Интерфейс должен быть адаптируемым под нужды и предпочтения пользователя.
\begin{itemize}
    \item Настройка частоты и типа получаемых уведомлений.
    \item Возможность отключения редко используемых подтверждений для опытных пользователей.
\end{itemize}

\item \textbf{Обучаемость (Learnability)}. Диалог должен способствовать изучению системы.
\begin{itemize}
    \item Ясные сообщения об ошибках являются инструментом обучения пользователя правильной работе с системой.
    \item Последовательная логика уведомлений и подтверждений помогает сформировать корректную ментальную модель.
\end{itemize}
\end{itemize}

\subsection{ГОСТ Р ИСО 9241-112: Принципы представления информации}

Эта часть стандарта фокусируется на эргономических принципах представления информации, включая текст, графику, звук и их композицию, что непосредственно касается визуального и смыслового дизайна изучаемых элементов.

\begin{itemize}
\item \textbf{Ясность (Clarity)}. Информация должна быть быстро и точно воспринимаема.
\begin{itemize}
\item Использование кратких, простых формулировок в сообщениях.
\item Отсутствие технического жаргона, непонятных кодов в тексте ошибок.
\end{itemize}
\item \textbf{Различимость (Discriminability)}. Представленная информация должна быть четко различимой.
\begin{itemize}
    \item Визуальное разделение типов сообщений (ошибка, предупреждение, информация) с помощью цвета, иконок и заголовков.
    \item Достаточный контраст и размер шрифта для комфортного чтения.
\end{itemize}

\item \textbf{Лаконичность (Conciseness)}. Информация должна быть представлена в компактной форме, без избыточных деталей.
\begin{itemize}
    \item Предоставление только необходимой для принятия решения информации в диалогах подтверждения.
    \item Возможность получить дополнительные детали об ошибке по запросу (например, через ссылку «Подробнее»).
\end{itemize}

\item \textbf{Последовательность (Consistency)}. Единообразие в представлении информации внутри системы.
\begin{itemize}
    \item Одинаковый стиль, терминология и расположение элементов для всех уведомлений, ошибок и подтверждений.
    \item Согласованность с общесистемными и платформенными гайдлайнами.
\end{itemize}

\item \textbf{Распознаваемость (Detectability)}. Внимание пользователя должно быть направлено на важную информацию, когда это необходимо.
\begin{itemize}
    \item Использование модальных окон для критически важных подтверждений и ошибок, требующих немедленного внимания.
    \item Визуальное выделение области с новым уведомлением.
\end{itemize}

\item \textbf{Читаемость (Legibility)}. Текстовая информация должна быть легко читаемой.
\begin{itemize}
    \item Использование типографики, обеспечивающей удобочитаемость (шрифт, интерлиньяж, длина строки).
\end{itemize}
\end{itemize}

Таким образом, ГОСТ Р ИСО 9241 предоставляет не разрозненный набор правил, а целостную, взаимосвязанную систему принципов. Проектирование ошибок, уведомлений и подтверждений в соответствии с этими принципами (в частности, 110-й и 112-й частями) обеспечивает создание интерфейса, который является не только функциональным, но и эффективным, безопасным и комфортным для пользователя, что напрямую соотносится с изначальным определением качества человеко-машинного интерфейса.

\newpage
\section{Сообщения об ошибках}

Сообщения об ошибках представляют собой важнейший компонент обратной связи в человеко-машинном интерфейсе. Их основная функция — информировать пользователя о возникновении проблемного состояния, которое препятствует нормальному выполнению задачи, и предоставить информацию для её разрешения. Эффективное сообщение об ошибке должно не только констатировать факт сбоя, но и объяснять его причину на понятном пользователю языке, предлагать конкретные, выполнимые действия для исправления ситуации и сохранять ощущение контроля у пользователя над системой.

\subsection{Классификация ошибок в человеко-машинном взаимодействии}

Ошибки, возникающие в процессе взаимодействия, можно классифицировать по их источнику и характеру:

\begin{itemize}
\item \textbf{Ошибки пользователя (Slips Mistakes)}:
\begin{itemize}
\item \textit{Опечатки и неверный ввод данных}: Например, ввод букв в поле, предназначенное только для цифр.
\item \textit{Ошибки по невнимательности (Slips)}: Непреднамеренные действия, такие как нажатие не той кнопки.
\item \textit{Ошибки понимания (Mistakes)}: Неправильные действия, вызванные неверной ментальной моделью работы системы.
\end{itemize}
\item \textbf{Ошибки системы (System Errors)}:
\begin{itemize}
    \item \textit{Внутренние сбои}: Отказ аппаратного обеспечения, программные исключения, недоступность сервера.
    \item \textit{Ошибки логики}: Неверные результаты вычислений, сбои в алгоритмах.
\end{itemize}

\item \textbf{Ошибки взаимодействия (Interaction Errors)}:
\begin{itemize}
    \item \textit{Ошибки валидации}: Данные не соответствуют бизнес-правилам (например, указана дата рождения из будущего).
    \item \textit{Ошибки состояния}: Попытка выполнить действие, недоступное в текущем состоянии системы (например, отправить заказ с пустой корзиной).
    \item \textit{Ошибки доступа}: Отсутствие необходимых прав для выполнения операции.
\end{itemize}
\end{itemize}

\subsection{Принципы оформления сообщений об ошибках в соответствии с нормативными требованиями (ГОСТ)}

При проектировании сообщений для критически важных систем, систем государственного назначения или в рамках строгой стандартизации часто опираются на принципы, изложенные в нормативных документах, таких как ГОСТ. В контексте эргономики и человеко-машинного интерфейса ключевым является \textbf{ГОСТ Р ИСО 9241-110-2009 «Эргономика взаимодействия человек-система. Принципы организации диалога»}. Общие требования к сообщениям об ошибках можно сформулировать следующим образом:

\begin{enumerate}
\item \textbf{Ясность и понятность}. Сообщение должно быть написано на языке пользователя, избегая технического жаргона и кодов.
\item \textbf{Конструктивность}. Сообщение должно чётко указывать, что произошло, почему это проблема и как её исправить.
\item \textbf{Краткость и точность}. Текст должен быть лаконичным, но информационно насыщенным.
\item \textbf{Вежливый и нейтральный тон}. Сообщение не должно обвинять пользователя, использовать негативно окрашенные слова («катастрофическая ошибка», «фатальный сбой»).
\item \textbf{Своевременность}. Ошибка должна сообщаться сразу после её возникновения, желательно в контексте места выполнения действия.
\item \textbf{Единообразие}. Формат, стиль и расположение сообщений должны быть единообразными во всей системе.
\end{enumerate}

\noindent\textbf{Примеры текстов сообщений об ошибках в формальном стиле:}

\begin{verbatim}
// Пример 1: Ошибка валидации данных (поле "ИНН")
[Ошибка ввода]
Введенное значение в поле "ИНН" некорректно.
ИНН физического лица должен состоять из 12 цифр.
Пожалуйста, проверьте данные и введите значение again.

// Пример 2: Ошибка состояния системы
[Операция недоступна]
Невозможно сформировать отчет за выбранный период.
Данные за указанные даты находятся в процессе обработки.
Пожалуйста, выберите другой период или повторите попытку позже.

// Пример 3: Системная ошибка доступа
[Отказано в доступе]
У вашей учетной записи недостаточно прав для выполнения
операции "Удаление пользователя".
Для получения доступа обратитесь к системному администратору.

// Пример 4: Ошибка соединения (системный сбой)
[Потеряно соединение с сервером]
Не удалось установить связь с основным сервером данных.
Причина: превышено время ожидания ответа.
Рекомендуемые действия:

Проверьте параметры сетевого подключения.
Убедитесь, что сервер находится в рабочем состоянии.
Повторите попытку через 2-3 минуты.
[Код ошибки: NET-408]
\end{verbatim}

\newpage
\section{Уведомления}

Уведомления в человеко-машинном интерфейсе представляют собой информационные сообщения, предназначенные для информирования пользователя о нормальных (не аварийных) событиях, изменениях состояния системы или завершении фоновых процессов. В отличие от ошибок, уведомления не требуют немедленного обязательного реагирования для продолжения работы, однако их своевременность и наглядность критически важны для поддержания пользователя в контексте выполняемой задачи и обеспечения прозрачности работы системы. Основная функция уведомлений — предоставление релевантной обратной связи без прерывания основного потока деятельности, если это не обусловлено важностью события.

\subsection{Классификация уведомлений}

Уведомления могут быть классифицированы по нескольким основаниям: по степени важности, источнику возникновения, способу представления и требуемому отклику.

\begin{itemize}
\item \textbf{По степени важности и срочности}:
\begin{itemize}
\item \textit{Информационные}: Сообщают о нормальном завершении операции или текущем статусе (например, «Документ сохранен», «Соединение установлено»). Часто не требуют взаимодействия и исчезают автоматически.
\item \textit{Предупреждающие}: Информируют о потенциально нежелательных или нестандартных ситуациях, которые могут привести к проблемам в будущем (например, «Батарея разряжена», «На диске осталось менее 10% свободного места»). Требуют внимания пользователя, но не обязательно немедленного действия.
\item \textit{Важные (Приоритетные)}: Сообщают о значимых событиях, требующих подтверждения факта ознакомления (например, «Входящий запрос на изменение данных», «Запланированное обслуживание системы завтра»). Могут блокировать интерфейс до получения отклика (модальные).
\end{itemize}
\item \textbf{По способу представления}:
\begin{itemize}
    \item \textit{Модальные (блокирующие)}: Требуют немедленного взаимодействия и временно останавливают работу с основным интерфейсом. Применяются для критически важных подтверждений.
    \item \textit{Немодальные (неблокирующие)}: Отображаются поверх основного интерфейса или в специальной области (тосты, панели уведомлений) и позволяют продолжить работу. Могут иметь таймер автоматического скрытия.
    \item \textit{Встроенные в контекст}: Отображаются непосредственно рядом с элементом интерфейса, к которому относятся (например, подсказка о формате ввода в поле).
\end{itemize}

\item \textbf{По источнику}:
\begin{itemize}
    \item \textit{Системные}: Генерируются операционной системой или платформой (обновления, события безопасности).
    \item \textit{Прикладные}: Генерируются в рамках конкретного программного приложения (уведомление о новом сообщении, завершении расчета).
    \item \textit{Пользовательские}: Инициируются другими пользователями в многопользовательских системах (запросы, сообщения).
\end{itemize}
\end{itemize}

\subsection{Принципы оформления уведомлений в соответствии с нормативными требованиями (ГОСТ)}

При проектировании систем, особенно в государственном и промышленном секторах, важным ориентиром являются стандарты, регламентирующие представление информации. \textbf{ГОСТ 9142-90 «Ящики производственные деревянные. Технические условия»} напрямую не касается вопросов человеко-машинного интерфейса. Однако, в более широком контексте стандартизации, принципы, заложенные в подобных документах, могут быть метафорически экстраполированы на проектирование ЧМИ: стандартизация форм, четкость назначения, надежность и соответствие ожиданиям. Для уведомлений применимы общие эргономические принципы из \textbf{ГОСТ Р ИСО 9241-110-2009} и аналогичных документов:

\begin{enumerate}
\item \textbf{Соответствие задаче}. Уведомление должно предоставлять информацию, релевантную текущему или инициированному пользователем контексту.
\item \textbf{Своевременность}. Информация должна появляться в момент, когда она наиболее полезна для пользователя, без избыточных задержек.
\item \textbf{Ненавязчивость}. Важность и способ отображения (модальность) должны быть адекватны значимости события. Не следует прерывать пользователя маловажными сообщениями.
\item \textbf{Структурированность и краткость}. Информация должна быть легко воспринимаемой. Рекомендуется использовать четкую структуру: суть события, детали (при необходимости), доступные действия.
\item \textbf{Визуальная дифференциация}. Важность уведомления должна быть быстро определяема по его внешнему виду (цвет, иконка, область экрана). Например, информационные — синий/зеленый, предупреждения — желтый/оранжевый, важные — красный (с осторожностью).
\item \textbf{Управляемость}. Пользователь должен иметь возможность настроить частоту, тип и способ получения уведомлений, а также просмотреть историю (журнал) событий.
\end{enumerate}

\noindent\textbf{Примеры текстов уведомлений в формальном стиле:}

\begin{verbatim}
// Пример 1: Информационное уведомление (немодальное, автоскрытие)
[Информация]
Соединение с центральным сервером успешно восстановлено.
Синхронизация данных выполняется в фоновом режиме.

// Пример 2: Предупреждающее уведомление (немодальное)
[Внимание]
Запланированное техническое обслуживание системы
состоится 25.05.2023 с 03:00 до 05:00 (МСК).
В этот период возможны кратковременные перерывы в работе.
Рекомендуем сохранить результаты работы заранее.

// Пример 3: Важное уведомление (модальное, требующее подтверждения)
[Требуется ваше решение]
От пользователя "Иванов А.С." поступил запрос на предоставление
доступа к папке "Проект_Альфа" с правами "Редактирование".
• Предоставить доступ с выбранными правами.
• Предоставить доступ только для чтения.
• Отклонить запрос.
[Выберите действие]

// Пример 4: Статусное уведомление (встроенное в интерфейс)
[Статус операции]
Формирование ежемесячного отчета... [||||||------] 60%
Расчет выполняется в фоновом режиме.
По завершении вы получите уведомление.
\end{verbatim}

\newpage
\section{Подтверждения}

Подтверждение в человеко-машинном интерфейсе представляет собой диалоговое окно (или иной модальный элемент интерфейса), которое требует от пользователя явного ответа (согласия или отказа) на выполнение запрашиваемого системой действия. Основная функция подтверждений — предотвращение необратимых или потенциально опасных ошибок пользователя за счет создания дополнительного шага осмысления перед финальным исполнением команды. Этот механизм реализует принцип «защиты от дурака» (fool-proofing) и служит важным инструментом обеспечения контроля над критическими операциями, такими как удаление данных, завершение работы без сохранения или финансовые транзакции.

\subsection{Классификация подтверждений}

Подтверждения могут быть классифицированы по нескольким ключевым признакам: характеру подтверждаемого действия, степени критичности и архитектуре диалога.

\begin{itemize}
\item \textbf{По характеру действия}:
\begin{itemize}
\item \textit{Подтверждения деструктивных операций}: Запрос согласия на действия, приводящие к безвозвратному удалению или изменению данных (удаление файла, записи, отмена регистрации).
\item \textit{Подтверждения необратимых системных изменений}: Запрос на выполнение действий, меняющих конфигурацию системы или программного обеспечения (установка обновлений, изменение критических настроек, перезагрузка сервера).
\item \textit{Подтверждения транзакций и фиксации данных}: Запрос на окончательное подтверждение ввода или отправки информации, имеющей юридические или финансовые последствия (отправка заказа, подтверждение платежа, подписание документа).
\item \textit{Подтверждения выхода без сохранения}: Предупреждение о возможной потере данных при закрытии документа или приложения без сохранения изменений.
\end{itemize}
\item \textbf{По степени критичности и дизайну ответа}:
\begin{itemize}
    \item \textit{Симметричные (нейтральные)}: Предлагают равноценные варианты выбора (например, «ОК» и «Отмена»). Риск ошибочного выбора умеренный.
    \item \textit{Асимметричные (с акцентом)}: Визуально или семантически выделяют безопасный или рекомендуемый вариант (например, «Отмена» представлена как основная кнопка, а «Удалить» — как второстепенная и окрашенная в красный). Используется для операций с высоким риском.
\end{itemize}

\item \textbf{По архитектуре диалога}:
\begin{itemize}
    \item \textit{Модальные (блокирующие)}: Требуют обязательного ответа, блокируя взаимодействие с родительским окном. Являются стандартом для подтверждений.
    \item \textit{Инлайновые (встроенные)}: Элемент подтверждения появляется в контексте интерфейса без блокировки всего окна (например, дополнительная кнопка «Подтвердить» после выбора действия).
\end{itemize}
\end{itemize}

\subsection{Принципы оформления подтверждений в соответствии с нормативными требованиями (ГОСТ)}

При проектировании диалогов подтверждения необходимо руководствоваться общими эргономическими принципами организации диалога, закрепленными в национальных и международных стандартах. Ключевым документом является \textbf{ГОСТ Р ИСО 9241-110-2009 «Эргономика взаимодействия человек-система. Принципы организации диалога»}, а также \textbf{ГОСТ Р ИСО 9241-112-2016 «Эргономика взаимодействия человек-система. Принципы представления информации»}. Требования к подтверждениям можно свести к следующим пунктам:

\begin{enumerate}
\item \textbf{Уместность (Relevance)}. Подтверждение должно запрашиваться только для действительно критических, необратимых или неординарных действий. Чрезмерное использование подтверждений приводит к «замыливанию» внимания пользователя и автоматическому согласию без чтения.
\item \textbf{Ясность и однозначность}. Текст диалога должен четко и просто объяснять:
\begin{itemize}
\item \textit{Контекст}: Какое действие инициировано.
\item \textit{Последствие}: Что именно произойдет после подтверждения.
\item \textit{Объект}: На какие именно данные или объекты повлияет действие.
\end{itemize}
\item \textbf{Согласованность с действием}. Формулировка вопроса в диалоге должна соответствовать первоначальной команде пользователя (например, если нажата кнопка «Удалить», в диалоге должен быть вопрос об удалении, а не об «отмене»).
\item \textbf{Понятные и логичные варианты ответа}. Надписи на кнопках должны прямо соответствовать заданному вопросу. Предпочтительно использовать глаголы действия («Удалить», «Сохранить», «Отменить»), а не абстрактные «Да»/«Нет». Более опасный вариант должен быть визуально менее заметным или требовать дополнительного действия (например, двойного клика).
\item \textbf{Обеспечение контроля у пользователя}. Диалог должен предоставлять явную возможность безопасного выхода из операции (кнопка «Отмена» или «Нет»). Фокус ввода по умолчанию должен быть установлен на безопасном варианте.
\item \textbf{Минимизация модальности}. Там, где это допустимо по уровню риска, следует рассмотреть возможность использования немодальных или отменяемых (undo) альтернатив вместо жесткого диалога подтверждения.
\end{enumerate}

\noindent\textbf{Примеры текстов диалогов подтверждения в формальном стиле:}

\begin{verbatim}
// Пример 1: Подтверждение деструктивной операции
[Подтверждение удаления]
Вы уверены, что хотите безвозвратно удалить проект
"Техническое задание 2025"?
Это действие нельзя будет отменить.
Все связанные файлы и записи будут удалены из системы.
[ Удалить проект ] [ Отменить ]

// Пример 2: Подтверждение системного изменения
[Изменение сетевых настроек]
Применение новых настроек IP-адресации приведет к
разрыву текущего сетевого подключения.
Возобновление работы может занять до 60 секунд.
Подтвердите внесение изменений в конфигурацию.
[ Применить настройки ] [ Отложить ]

// Пример 3: Подтверждение транзакции
[Фиксация финансовой операции]
Подтвердите проведение платежа.
Сумма к списанию: 15 750,00 руб.
Получатель: ООО "ТехноСнаб"
Назначение: Оборудование по договору №345 от 12.10.2025
Списание будет произведено с карты *2874.
[ Подтвердить и оплатить ] [ Вернуться к проверке ]

// Пример 4: Подтверждение выхода без сохранения
[Сохранение документа]
Документ "Отчет_Промежуточный.docx" содержит несохраненные
изменения.
Сохранить изменения перед закрытием?
[ Сохранить ] [ Не сохранять ] [ Отменить закрытие ]
\end{verbatim}

\newpage
\section*{Заключение}
\addcontentsline{toc}{section}{Заключение}
В ходе проведенного анализа роли и принципов проектирования сообщений об ошибках, уведомлений и подтверждений было установлено, что данные элементы представляют собой критически важный триадический механизм обратной связи в человеко-машинном интерфейсе. Их комплексная и сбалансированная реализация непосредственно определяет выполнение ключевых функций ЧМИ: обеспечение контроля пользователя над системой, поддержание его в продуктивном состоянии и предотвращение ошибочных действий.

\textbf{Сообщения об ошибках} выступают как реактивный, но конструктивный механизм, предназначенный для восстановления работоспособности системы после сбоя. Их эффективность определяется способностью не просто констатировать проблему, а направлять пользователя к её разрешению, тем самым снижая фрустрацию и поддерживая обучаемость системы. \textbf{Уведомления}, в свою очередь, выполняют проактивную и информативную функцию, обеспечивая прозрачность работы системы и информируя о нормальных, но значимых событиях, не нарушая при этом основной поток деятельности. \textbf{Подтверждения} играют роль превентивного защитного барьера, создавая осознанную паузу перед выполнением потенциально критических операций и тем самым повышая надежность и безопасность взаимодействия.

Эти три компонента не существуют изолированно, а образуют единый контур диалога между пользователем и системой. Уведомление может предвосхитить возможную ошибку, подтверждение — предотвратить её, а хорошо составленное сообщение об ошибке — исправить последствия. Проектирование данного контура должно основываться на фундаментальных эргономических принципах, систематизированных в стандартах серии \textbf{ГОСТ Р ИСО 9241}, в частности, принципах пригодности для задачи, управляемости, ошибкоустойчивости и соответствия ожиданиям пользователя.

Таким образом, качество человеко-машинного интерфейса в значительной мере зависит от того, насколько грамотно, последовательно и в соответствии с нормативными требованиями реализованы механизмы обратной связи. Интегрированный подход к проектированию ошибок, уведомлений и подтверждений, учитывающий их взаимосвязь и подчиненный единым принципам организации диалога, является необходимым условием создания эффективных, безопасных и ориентированных на пользователя систем. Именно такой подход позволяет трансформировать формальное взаимодействие человека с машиной в содержательный и эффективный диалог, что и составляет конечную цель эргономичного проектирования интерфейсов.

\newpage
\addcontentsline{toc}{section}{Список литературы}
\begin{thebibliography}{9}
    \bibitem{cooper}
    Алан Купер. \emph{Об интерфейсе. Основы проектирования взаимодействия}. --- 2-е изд. --- М.: ИМВ, 2009. — 688 с.
    \end{thebibliography}

\end{document}