\documentclass[12pt]{article}
\usepackage[utf8]{inputenc}
\usepackage[T2A]{fontenc}
\usepackage[russian]{babel}
\usepackage[a4paper,margin=2.2cm]{geometry}
\usepackage{enumitem}
\usepackage{hyperref}

\title{Тест из 25 вопросов\\\large Стадии проектирования автоматизированных информационных систем}
\author{Салимли Айзек}
\date{}

\begin{document}
\maketitle

\begin{enumerate}[leftmargin=*,itemsep=0.9em]

\item На какой стадии впервые фиксируется \emph{полная} совокупность функциональных и нефункциональных требований в форме, пригодной для приёмки?
  \begin{enumerate}[label=\alph*)]
    \item Содержательное описание предметной области
    \item Формализация (математическая модель)
    \item \textbf{Разработка и согласование ТЗ на систему и подсистемы}
    \item Пилотная эксплуатация
  \end{enumerate}

\item Что является корректным критерием завершения стадии \emph{содержательного описания}?
  \begin{enumerate}[label=\alph*)]
    \item Наличие исходного кода ключевых модулей
    \item \textbf{Согласованный глоссарий и модели бизнес-процессов, подтверждённые заказчиком}
    \item Проведение нагрузочного тестирования
    \item Утверждённый план сопровождения
  \end{enumerate}

\item Какой результат \emph{формализации} обеспечивает проверяемость непротиворечивости требований?
  \begin{enumerate}[label=\alph*)]
    \item Диаграммы пользовательских интерфейсов (UI mockups)
    \item \textbf{Формальная модель (например, онтология/сети Петри/AADL) с правилами вывода/симуляцией}
    \item Проект плана внедрения с бюджетом
    \item Паспорт ИБ-мероприятий
  \end{enumerate}

\item Что является основным назначением пилотного проекта перед промышленным внедрением?
  \begin{enumerate}[label=\alph*)]
    \item Сокращение документации
    \item Замена всех интеграций на временные
    \item \textbf{Проверка гипотез на реальных данных и уточнение рисков/метрик качества}
    \item Переписывание архитектуры с нуля
  \end{enumerate}

\item Как обеспечить устойчивость к изменениям предметной области при переходах между стадиями?
  \begin{enumerate}[label=\alph*)]
    \item Жёсткое кодирование правил в интерфейсе
    \item Отказ от версионирования
    \item \textbf{Трассируемость «требование$\rightarrow$модель$\rightarrow$архитектура$\rightarrow$реализация$\rightarrow$тест»}
    \item Хранение требований только в почте
  \end{enumerate}

\item Что из ниже перечисленного \emph{непосредственно} снижает стоимость правок схем данных на поздних стадиях?
  \begin{enumerate}[label=\alph*)]
    \item Фиксация требований только в ТЗ
    \item Ранний выбор СУБД
    \item \textbf{Метауровень: метаданные/металогика/мета-интерфейсы и генерация артефактов}
    \item Увеличение численности команды
  \end{enumerate}

\item Какой артефакт связывает содержательное описание с формализованной моделью?
  \begin{enumerate}[label=\alph*)]
    \item План закупок
    \item \textbf{Карты сценариев/варианты использования с формальными пред-/постусловиями}
    \item Политика резервного копирования
    \item Руководство пользователя
  \end{enumerate}

\item Что является корректным критерием готовности ТЗ к разработке?
  \begin{enumerate}[label=\alph*)]
    \item Указаны только архитектурные диаграммы
    \item \textbf{Определены границы системы, интерфейсы с внешними ИС, нефункциональные требования и критерии приёмки}
    \item Утверждён только бюджет
    \item Есть прототип UI без описания API
  \end{enumerate}

\item На какой стадии целесообразно провести моделирование очередей/нагрузок для оценки производительности?
  \begin{enumerate}[label=\alph*)]
    \item Только после внедрения
    \item \textbf{На стадии формализации/проектирования с использованием исполняемых моделей}
    \item На стадии сопровождения
    \item На этапе закупки серверов, без модели
  \end{enumerate}

\item Какая практика минимизирует разрыв между пилотом и промышленным внедрением?
  \begin{enumerate}[label=\alph*)]
    \item Отдельные, несовместимые ветки кода
    \item \textbf{Единый конвейер CI/CD и миграции БД, применимые и на пилоте, и в проде}
    \item Запрет автоматического тестирования
    \item Хардкод тестовых заглушек в основной ветке
  \end{enumerate}

\item Что корректно характеризует переход от классического подхода к метамодельному?
  \begin{enumerate}[label=\alph*)]
    \item Отказ от формальной модели
    \item \textbf{Сдвиг фокуса с «ручной» разработки артефактов к описанию метамодели и генерации}
    \item Увеличение числа нестандартизованных интерфейсов
    \item Игнорирование требований к качеству
  \end{enumerate}

\item Какой показатель следует зафиксировать ещё в ТЗ для объективной приёмки после внедрения?
  \begin{enumerate}[label=\alph*)]
    \item Цветовая схема интерфейса
    \item \textbf{Целевые SLA/SLO: время отклика, доступность, ошибки на транзакцию}
    \item Фамилии всех пользователей
    \item Набор скриншотов
  \end{enumerate}

\item Что из перечисленного относится к проверке \emph{вал\textbf{и}дации} на стадии пилота?
  \begin{enumerate}[label=\alph*)]
    \item \textbf{Подтверждение, что система решает бизнес-задачу на реальных процессах}
    \item Проверка соответствия коду стандартам оформления
    \item Анализ покрытия модульными тестами
    \item Линтинг SQL-скриптов
  \end{enumerate}

\item Какая роль должна утвердить глоссарий и границы системы до формализации?
  \begin{enumerate}[label=\alph*)]
    \item Архитектор без участия бизнеса
    \item Разработчик интерфейсов
    \item \textbf{Заказчик (владелец процессов) совместно с системным аналитиком}
    \item Команда сопровождения
  \end{enumerate}

\item Какой набор артефактов оптимален для передачи в разработку после формализации?
  \begin{enumerate}[label=\alph*)]
    \item Набор презентаций и скриншотов
    \item \textbf{Формальная модель, спецификация интерфейсов, модель данных, критерии тестирования}
    \item Только диаграммы классов
    \item Только текст ТЗ без интерфейсов
  \end{enumerate}

\item При интеграции с внешними ИС на какой стадии нужно зафиксировать контракты API и версии?
  \begin{enumerate}[label=\alph*)]
    \item На сопровождении, по факту ошибок
    \item На этапе пилота, не документируя
    \item \textbf{В ТЗ/проектировании с управлением конфигурацией и версионированием}
    \item Во время тестовой эксплуатации без контроля изменений
  \end{enumerate}

\item Что является корректным выходом стадии «внедрение»?
  \begin{enumerate}[label=\alph*)]
    \item Черновые инструкции для отдельных пользователей
    \item \textbf{Принятая система по актам приёмки с выполненными критериями и обучением персонала}
    \item Только запуск сервера
    \item Устные договорённости о сроках доработок
  \end{enumerate}

\item Какой риск наиболее типичен при переходе от содержательного описания к формальной модели?
  \begin{enumerate}[label=\alph*)]
    \item Снижение прозрачности требований
    \item \textbf{Потеря смысла терминов при некорректном мэппинге в формальные конструкции}
    \item Избыточная автоматизация тестов
    \item Недостаток серверных мощностей
  \end{enumerate}

\item Для снижения стоимости исправлений где рационально сосредоточить основные проверки согласованности?
  \begin{enumerate}[label=\alph*)]
    \item \textbf{На стадии формализации/проектирования: статический анализ и симуляция моделей}
    \item Только в нагрузочных тестах на проде
    \item На этапе сопровождения
    \item В визуальном дизайне UI
  \end{enumerate}

\item Какая метрика подтверждает готовность к промышленной эксплуатации после пилота?
  \begin{enumerate}[label=\alph*)]
    \item Количество прототипов UI
    \item \textbf{Стабильные метрики надёжности/производительности на пилотных данных, совпадающих по профилю с боевыми}
    \item Число страниц ТЗ
    \item Количество задействованных разработчиков
  \end{enumerate}

\item Как обеспечить переносимость результатов формализации в реализацию?
  \begin{enumerate}[label=\alph*)]
    \item Свободное толкование формальных артефактов программистами
    \item \textbf{Генерация схем/конфигураций/каркасов из метамодели + проверки соответствия (model-to-text/consistency checks)}
    \item Хранение модели в закрытых форматах без экспорта
    \item Удаление формальных ограничений при кодировании
  \end{enumerate}

\item Что должно быть определено до начала пилота в части данных?
  \begin{enumerate}[label=\alph*)]
    \item \textbf{Стратегия миграции и обезличивания/маскирования, эталонные наборы тестовых данных}
    \item Только дашборды
    \item Эскизы экранных форм
    \item План командировок
  \end{enumerate}

\item Какой признак указывает, что стадия «ТЗ» выполнена формально и несёт риски на внедрении?
  \begin{enumerate}[label=\alph*)]
    \item \textbf{Отсутствуют измеримые критерии приёмки для нефункциональных требований}
    \item В ТЗ описаны интерфейсы и ограничения
    \item Есть матрица трассируемости
    \item Определён регламент изменения требований
  \end{enumerate}

\item Какой подход корректен для управления изменениями между пилотом и внедрением?
  \begin{enumerate}[label=\alph*)]
    \item Вносить правки напрямую в прод без версий
    \item \textbf{Процедура изменения требований (change control), обновление ТЗ/моделей и повторная приёмка}
    \item Фиксировать изменения устно
    \item Отключить автоматические тесты на время
  \end{enumerate}

\item Какой артефакт завершает цикл «требование$\rightarrow$реализация» на стадии внедрения?
  \begin{enumerate}[label=\alph*)]
    \item Техническое задание в черновике
    \item \textbf{Протокол приёмочно-сдаточных испытаний с ссылками на требования и результаты тестов}
    \item Набор макетов интерфейса
    \item План отпусков команды
  \end{enumerate}

\end{enumerate}

\end{document}
