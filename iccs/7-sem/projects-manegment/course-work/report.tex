\documentclass[areasetadvanced]{scrartcl}

\usepackage[utf8]{inputenc}
\usepackage[T2A]{fontenc}
\usepackage[english,russian]{babel}
\usepackage{xcolor}

\usepackage[footskip=1cm,left=25mm, right=15mm, top=20mm, bottom=20mm]{geometry}
\usepackage{setspace}
\usepackage{amsmath, amssymb} 
\usepackage{graphicx}
\usepackage{tikz}
\usetikzlibrary{arrows.meta}
\usepackage{float}
\usepackage{dashrule}
\usepackage{fancyhdr} 
\usepackage{hyperref} 
\usepackage{parskip}
\usepackage{textcomp, enumitem}
\usepackage{indentfirst}
\usepackage{graphicx}
\usepackage{algorithm}
\usepackage{algpseudocode}
\usepackage{array} 
\usepackage{geometry}
\usepackage{afterpage}
\usepackage{minted}
\setcounter{secnumdepth}{3} 
\setcounter{tocdepth}{3}    
\usepackage{listings} 
\usepackage{booktabs}
\usepackage{paracol} % параллельные колонки (левая/правая)

\newcommand{\icon}[1]{\includegraphics[height=1.2em]{#1}}

\tikzstyle{block} = [rectangle, rounded corners, minimum width=3cm, minimum height=1cm, text centered, draw=black, fill=lightgray]

\setkomafont{sectioning}{\normalfont\bfseries} 
\setkomafont{section}{\normalfont\Large\bfseries}
\setkomafont{subsection}{\normalfont\large\bfseries}
\setkomafont{subsubsection}{\normalfont\large\bfseries}
\setkomafont{paragraph}{\normalfont\large\bfseries} 
\newcommand{\twosideheading}[2]{%
  \noindent
  \begin{minipage}[t]{0.5\textwidth}\raggedright\small #1\end{minipage}%
  \begin{minipage}[t]{0.5\textwidth}\raggedleft\Large\bfseries #2\end{minipage}\\[-0.2em]
  \hrule
  \vspace{0.8em}
}
\lstset{
  language=Haskell,
  basicstyle=\ttfamily\small,
  keywordstyle=\color{blue}\bfseries,
  stringstyle=\color{red},
  commentstyle=\color{green!70!black},
  numbers=left,
  numberstyle=\tiny,
  stepnumber=1,
  numbersep=10pt,
  showstringspaces=false,
  breaklines=true,
  frame=single
}

\lstdefinelanguage{Lua}{
    keywords={function, end, if, then, else, elseif, for, while, do, repeat, until, break, return, local, and, or, not, true, false, nil},
    keywordstyle=\color{blue}\bfseries,
    stringstyle=\color{red},
    commentstyle=\color{green!70!black},
    morestring=[s]{"}{"},
    morestring=[s]{'}{'},
    morecomment=[l]{--},
    morecomment=[s]{--[[}{]]},
    basicstyle=\ttfamily\small,
    numbers=left,
    numberstyle=\tiny,
    stepnumber=1,
    numbersep=10pt,
    showstringspaces=false,
    breaklines=true,
    frame=single
}

\lstdefinestyle{py}{
    language=Python,
    basicstyle=\ttfamily\small,
    keywordstyle=\color{blue}\bfseries,
    stringstyle=\color{red},
    commentstyle=\color{green!70!black},
    numbers=left,
    numberstyle=\tiny,
    stepnumber=1,
    numbersep=10pt,
    showstringspaces=false,
    breaklines=true,
    frame=single
}

\setlength{\parindent}{1.25cm}
\setcounter{tocdepth}{3}
\begin{document}
\sloppy
	\thispagestyle{empty}
	\begin{center}
		\large{МИНОБРНАУКИ РОССИИ} \par
		\vspace{0.3cm}
		\normalsize
		{ФЕДЕРАЛЬНОЕ ГОСУДАРСТВЕННОЕ АВТОНОМНОЕ ОБРАЗОВАТЕЛЬНОЕ УЧРЕЖДЕНИЕ ВЫСШЕГО ОБРАЗОВАНИЯ} \par
		\vspace{0.3cm}
		\textbf{\guillemotleft САНКТ-ПЕТЕРБУРГСКИЙ ПОЛИТЕХНИЧЕСКИЙ}
		\textbf{УНИВЕРСИТЕТ ПЕТРА ВЕЛИКОГО\guillemotright} \par
		\vspace{0.3cm}
		{Институт компьютерных наук и кибербезопасности}\par
		{Высшая школа технологий искусственного интеллекта}\par
	\end{center}
	\vfill
	\begin{center}
		{\large Отчёт по дисциплине \guillemotleft Управление проектами\guillemotright}\par
		\vspace{1cm}
		\Huge КУРСОВАЯ РАБОТА\par
		\vspace{0.5cm}
		{\huge \guillemotleft Управление проектом: Распознование объектов в режиме реального времени\guillemotright }\par
	\end{center}
	\vfill
	\begin{flushleft}
		Студент: \hspace{1.8cm} \rule[0pt]{2.5cm}{0.5pt}\hfill Салимли Айзек Мухтар Оглы\par
		\vspace{1.5cm}
		Преподаватель: \hspace{0.55cm} \rule[0pt]{2.5cm}{0.5pt}\hfill  Большаков Александр Афанасьевич
	\end{flushleft}
	\vspace{0.5cm}
	\begin{flushright}
		\guillemotleft \rule[0pt]{0.8cm}{0.5pt}\guillemotright \rule[0pt]{2cm}{0.5pt} 20\rule[0pt]{0.5cm}{0.5pt} г.
	\end{flushright}
	\vfill
	\begin{center}
		Санкт-Петербург, 2025
	\end{center}
	\newpage
	\tableofcontents
	\newpage

\section*{Введение}
\addcontentsline{toc}{section}{Введение}
Данная работа посвящена анализу методов управления проектами и инструментальному обеспечению для их реализации. В рамках исследования выбирается актуальная тематика проекта, после чего производится описание применения наиболее подходящего метода управления с учётом поставленных задач. Практическая часть работы включает в себя частичную реализацию выбранного метода. 

\textcolor{red}{\textbf{Полная реализация самого проекта не входит в задачи данного исследования.}}

\newpage
\section{Постановка задачи}
Актуальность данной задачи обусловлена активным развитием высокоскоростных трамваев в России, в частности в Санкт-Петербурге, где планируются новые маршруты с высокоскоростными трамваями. В этих условиях обнаружение посторонних объектов на трамвайных путях в режиме реального времени становится важной задачей для обеспечения безопасности и бесперебойности движения. Повышение скоростного режима многократно увеличивает риски и последствия от столкновений с препятствиями.

Существующие системы видеонаблюдения, требующие постоянного внимания оператора, могут быть не эффективны, так как сопряжены с риском человеческой ошибки из-за усталости и снижения концентрации. Это может приводить к несвоевременному обнаружению препятствий, таких как автомобили, упавшие деревья или крупный мусор, что чревато аварийными ситуациями, сбоями в графике движения высокоскоростных линий и значительным материальным ущербом.

В связи с этим была поставлена задача разработать систему компьютерного зрения, которая позволит в реальном времени автоматически обнаруживать и классифицировать посторонние объекты на трамвайных путях.

Система должна выполнять следующие функции:
\begin{itemize}
    \item Принимать на вход видеопоток с установленных на трамваях или инфраструктуре камер в режиме реального времени;
    \item Обрабатывать каждый кадр видео с использованием алгоритмов компьютерного зрения и предварительно обученной нейронной сети для обнаружения объектов;
    \item Детектировать и классифицировать посторонние объекты с высокой точностью и минимальным количеством ложных срабатываний;
    \item Определять местоположение обнаруженного объекта относительно трамвайных путей (путем передачи данных о расположении камеры);
    \item В случае обнаружения потенциально опасного объекта немедленно формировать и передавать сигнал оповещения в диспетчерский центр и водителю трамвая с указанием типа объекта и его локации;
    \item Визуализировать результаты детекции для оператора (совершить снимок) для удобства мониторинга и анализа;
    \item Обеспечивать возможность ручной разметки и дообучения модели на новых типах объектов для повышения точности и адаптации под конкретные условия эксплуатации.
\end{itemize}
Система должна быть спроектирована как надежное и отказоустойчивое решение; иметь возможность обработки видеопотоков с нескольких камер одновременно и функционировать с минимальной задержкой. Архитектура должна быть микросервисной, что позволит в будущем легко заменять или обновлять детектирующие модели, добавлять новые источники видео или интегрироваться с внешними системами управления без кардинального изменения исходного кода.

\newpage
\section{Подходы к управлению проектами}
\subsection{Классификация}

Всё многообразие подходов к управлению проектами может быть условно классифицировано на три основные группы, различающиеся принципами планирования, выполнения и контроля:

\begin{enumerate}
    \item Классические (предиктивные, каскадные) методологии
    \item Гибкие (адаптивные, итеративные) методологии  
    \item Гибридные методологии
\end{enumerate}

Рассмотрим подробно каждую из указанных групп.

\subsection{Классические (предиктивные) методологии}

Классические методологии основаны на линейном и последовательном подходе к реализации проекта. Ключевой принцип заключается в детальном планировании всех этапов работ на начальной стадии и строгом соблюдении утверждённого плана.

\subsubsection{Водопадная модель (Waterfall)}

Наиболее известным представителем данного класса является водопадная модель. Проект разделяется на строго определённые, последовательные фазы, где переход к следующему этапу возможен только после полного завершения предыдущего.

\begin{itemize}
    \item \textbf{Фазы:} Анализ требований → Проектирование → Разработка → Тестирование → Внедрение → Поддержка
    \item \textbf{Особенности:}
    \begin{itemize}
        \item Жёсткое планирование: Все требования к продукту фиксируются на начальном этапе и не подлежат существенным изменениям
        \item Обширная документация: Каждый этап сопровождается созданием подробной проектной документации
        \item Линейный прогресс: Возврат на предыдущие этапы для внесения изменений требует формального процесса управления изменениями
    \end{itemize}
    \item \textbf{Преимущества:}
    \begin{itemize}
        \item Простота и понятность структуры управления
        \item Предсказуемость сроков и бюджета при стабильных требованиях
        \item Строгий контроль над ходом проекта
    \end{itemize}
    \item \textbf{Недостатки:}
    \begin{itemize}
        \item Низкая гибкость и медленная реакция на изменения
        \item Позднее выявление ошибок, что увеличивает стоимость их исправления
        \item Заказчик получает возможность оценить продукт только на завершающей стадии
    \end{itemize}
\end{itemize}

\textbf{Область применения:} Водопадная модель эффективна для проектов с чётко определёнными, неизменными требованиями и предсказуемой технологической средой, например в строительстве или при создании типовых систем с минимальной инновационной составляющей.

\subsubsection{Стандарт PMBOK}

Хотя PMBOK (Project Management Body of Knowledge) не является методологией в строгом смысле, он представляет собой свод знаний и лучших практик в области управления проектами. Стандарт описывает 5 групп процессов (инициация, планирование, исполнение, мониторинг и контроль, завершение) и 10 областей знаний (управление интеграцией, содержанием, сроками, стоимостью, качеством и др.). PMBOK часто служит основой для построения классических систем управления в крупных организациях.

\subsection{Гибкие (адаптивные) методологии}

Гибкие методологии возникли как ответ на ограничения классических подходов, особенно в сфере разработки программного обеспечения. Они основаны на итеративной разработке, тесном взаимодействии с заказчиком и готовности к изменениям. Основные принципы изложены в «Манифесте гибкой разработки программного обеспечения».

\subsubsection{Scrum}

Scrum — это фреймворк, ориентированный на командную работу, постоянное обучение и самоорганизацию при решении задач.

\begin{itemize}
    \item \textbf{Ключевые элементы:}
    \begin{itemize}
        \item Роли: Владелец Продукта, Скрам-мастер, Команда Разработки
        \item Артефакты: Бэклог Продукта, Бэклог Спринта, Инкремент продукта
        \item События: Спринт, Планирование спринта, Ежедневный скрам, Обзор спринта, Ретроспектива спринта
    \end{itemize}
    \item \textbf{Преимущества:}
    \begin{itemize}
        \item Высокая адаптивность к изменениям требований
        \item Быстрая поставка работающих частей продукта
        \item Прозрачность процесса для всех участников
    \end{itemize}
    \item \textbf{Недостатки:}
    \begin{itemize}
        \item Сложность точного прогнозирования сроков и бюджета
        \item Требует высокой вовлечённости заказчика и квалификации команды
    \end{itemize}
\end{itemize}

\textbf{Область применения:} Scrum оптимален для инновационных проектов с нечёткими или изменяющимися требованиями, где конечный результат сложно предсказать заранее.

\subsubsection{Kanban}

Kanban — это метод, направленный на оптимизацию рабочих процессов и минимизацию незавершённой работы через визуализацию.

\begin{itemize}
    \item \textbf{Принципы:}
    \begin{itemize}
        \item Визуализация рабочего процесса с использованием Kanban-доски
        \item Ограничение количества задач в работе (WIP - Work in Progress)
        \item Управление и непрерывная оптимизация потока задач
    \end{itemize}
    \item \textbf{Преимущества:}
    \begin{itemize}
        \item Максимальная гибкость без фиксированных итераций
        \item Простота внедрения и понимания
        \item Фокус на непрерывной поставке ценности
    \end{itemize}
    \item \textbf{Недостатки:}
    \begin{itemize}
        \item Отсутствие чётких временных рамок может снижать дисциплину
        \item Не определяет конкретные роли и регулярные встречи
    \end{itemize}
\end{itemize}

\textbf{Область применения:} Kanban эффективен для проектов с постоянным потоком разнородных задач, таких как техническая поддержка или непрерывное совершенствование продуктов.

\subsection{Гибридные методологии}

Гибридные подходы интегрируют преимущества классических и гибких методологий, обеспечивая стратегический контроль и предсказуемость на высоком уровне при одновременном предоставлении операционной гибкости командам разработки.

\begin{itemize}
    \item \textbf{Пример реализации:}
    \begin{itemize}
        \item Верхний уровень (стратегический): Планирование на основе классических подходов с определением ключевых вех, бюджета и общих требований
        \item Нижний уровень (операционный): Реализация по гибким методологиям с выполнением работ короткими итерациями и адаптацией требований в рамках установленных вех
    \end{itemize}
    \item \textbf{Преимущества:}
    \begin{itemize}
        \item Сочетание предсказуемости и гибкости
        \item Удобство интеграции в крупные организации с устоявшимися процессами
    \end{itemize}
    \item \textbf{Недостатки:}
    \begin{itemize}
        \item Сложность согласования различных подходов и риск конфликтов между ними
        \item Требует высокой управленческой зрелости организации
    \end{itemize}
\end{itemize}
\begin{table}[H]
    \centering
    \caption{Сравнительная таблица методологий управления проектами}
    \label{tab:methodology_comparison}
    \begin{tabular}{|p{3cm}|p{3.5cm}|p{3.5cm}|p{3.5cm}|}
    \hline
    \textbf{Критерий} & \textbf{Классические (Waterfall)} & \textbf{Гибкие (Scrum/Kanban)} & \textbf{Гибридные} \\
    \hline
    Планирование & Детальное, на весь проект & Итеративное, краткосрочное & Высокоуровневое на старте, детальное итеративное \\
    \hline
    Гибкость & Низкая, изменения не приветствуются & Высокая, изменения являются частью процесса & Умеренная, гибкость в рамках этапов \\
    \hline
    Вовлеченность заказчика & Низкая (на этапах сбора требований и приемки) & Высокая и постоянная & Периодическая (на обзорах и ключевых точках) \\
    \hline
    Документация & Обширная и формализованная & Минимально необходимая & Комбинированная \\
    \hline
    Поставка ценности & В конце проекта & Постоянная, инкрементальная & По завершении крупных этапов (вех) \\
    \hline
    Лучше всего подходит для & Проектов с четкими и стабильными требованиями & Инновационных проектов с высокой неопределенностью & Крупных проектов в корпоративной среде \\
    \hline
    \end{tabular}
    \end{table}

\newpage
\section{Классификация программного обеспечения для управления проектами}
Практическая реализация рассмотренных ранее методологий управления проектами требует применения специализированного программного обеспечения. Современные инструменты проектного менеджмента позволяют автоматизировать рутинные задачи, обеспечивать видимость рабочих процессов, организовывать эффективное взаимодействие между участниками команды и формировать аналитическую отчетность для принятия обоснованных управленческих решений. 

Выбор конкретного программного решения определяется комплексом факторов, включающих принятую в проекте методологию, его масштаб и сложность, численность команды, требования к защите информации и специфику решаемых задач.

В рамках данного раздела представлен систематизированный обзор и классификация современных программных средств управления проектами, что создает методологическую основу для обоснованного выбора инструментария при реализации проекта по созданию системы интеллектуального поиска.

\subsection{Классификации программного обеспечения}

Программные средства управления проектами могут быть систематизированы по ряду ключевых критериев.

\subsubsection{По поддерживаемой методологии}

\begin{itemize}
    \item \textbf{Инструменты для классического подхода (Waterfall):} Сфокусированы на детальном планировании, построении иерархических структур работ (WBS) и контроле выполнения с использованием диаграмм Ганта. Пример: Microsoft Project.
    
    \item \textbf{Инструменты для гибкой разработки (Agile):} Обеспечивают поддержку итеративных методологий (Scrum, Kanban) через реализацию бэклогов, спринтов и канбан-досок. Пример: Jira Software, YouTrack.
    
    \item \textbf{Универсальные и гибридные платформы:} Комбинируют элементы классического и гибкого подходов, предоставляя возможность адаптации инструмента под специфические процессы команды. Пример: Asana, ClickUp.
\end{itemize}

\subsubsection{По модели развертывания}

\begin{itemize}
    \item \textbf{Облачные решения (SaaS):} Программное обеспечение предоставляется по подписке через веб-браузер. Не требует локальной установки и обслуживания. Большинство современных систем (Jira Cloud, Asana, Trello) используют эту модель.
    
    \item \textbf{Локальные решения (On-premise):} Установка и эксплуатация ПО на собственных серверах организации. Обеспечивает полный контроль над данными и безопасностью, но требует значительных ресурсов для администрирования. Пример: Jira Data Center, GitLab Self-Managed, Redmine.
\end{itemize}

\subsection{Анализ инструментальных средств по категориям}

\subsubsection{Инструменты для классического управления проектами}

\textbf{Microsoft Project} — отраслевой стандарт для предиктивного управления. Мощное решение для детального планирования, управления ресурсами и построения сложных расписаний. Ключевые функции включают создание диаграмм Ганта, расчет критического пути, управление бюджетами и распределение ресурсов. Несмотря на широкие возможности, инструмент отличается сложностью освоения и ограниченной гибкостью для проектов с высокой неопределенностью.

\subsubsection{Инструменты для Agile-разработки}

\textbf{Jira Software} — продукт компании Atlassian, признанный отраслевой стандарт для IT-команд. Обеспечивает комплексную поддержку Scrum (спринты, бэклоги, диаграммы сгорания) и Kanban (доски, WIP-лимиты). Основные преимущества — высокая степень кастомизации рабочих процессов и развитая экосистема интеграций.

\textbf{YouTrack} — решение от JetBrains, прямой конкурент Jira. Сочетает функции баг-трекера и системы управления Agile-проектами. Отличается развитой системой работы с задачами, ориентированной на разработчиков, включая интеллектуальные запросы и команды управления.

\textbf{Kaiten} — отечественная разработка с акцентом на визуализацию рабочих процессов. Позволяет объединять несколько досок в едином пространстве для отслеживания сквозных процессов между командами. Поддерживает как Kanban, так и Scrum.

\subsubsection{Универсальные платформы}

\textbf{Asana} — гибкий инструмент с поддержкой различных форматов представления данных: списки, доски, таймлайны и календари. Отличается простотой использования и развитыми средствами автоматизации, что делает его популярным в различных сферах деятельности.

\textbf{ClickUp} — позиционируется как единая платформа для организации работы. Объединяет функционал для управления задачами, документами, таблицами и коммуникациями. Высокая степень настройки позволяет адаптировать рабочее пространство под различные требования.

\subsubsection{DevOps-платформы}

\textbf{GitLab} — комплексное решение, охватывающее полный жизненный цикл разработки. Встроенные инструменты управления проектами включают доски задач, эпики, вехи и дорожные карты. Ключевое преимущество — глубокая интеграция управления задачами с репозиториями кода и процессами CI/CD.

\subsubsection{Решения с открытым исходным кодом}

\textbf{Redmine} — классическое Open Source решение на базе Ruby on Rails. Распространяется бесплатно и поддерживает установку на собственные серверы. Включает систему отслеживания задач, диаграммы Ганта, календарь и wiki-документирование. Отличается высокой гибкостью благодаря модульной архитектуре.

\begin{table}[H]
    \centering
    \caption{Сравнительный анализ ключевых инструментальных средств}
    \label{tab:tools_comparison}
    \begin{tabular}{|p{2.5cm}|p{2.2cm}|p{5cm}|p{4.5cm}|}
    \hline
    \textbf{Инструмент} & \textbf{Основная методология} & \textbf{Сильные стороны} & \textbf{Оптимально для} \\
    \hline
    MS Project & Классическая (Waterfall) & Детальное планирование ресурсов, сроков и бюджета; диаграммы Ганта & Крупных, формализованных проектов в строительстве, инжиниринге \\
    \hline
    Jira Software & Agile (Scrum, Kanban) & Глубокая кастомизация, мощные отчеты, огромная экосистема Atlassian & Профессиональных команд разработки ПО любого размера \\
    \hline
    YouTrack & Agile (Scrum, Kanban) & Фокус на разработчиках, умные запросы, интеграция с IDE от JetBrains & IT-команд, которые ищут мощную альтернативу Jira \\
    \hline
    Asana & Универсальная, гибридная & Простота использования, разные виды отображения, автоматизация & Бизнес-команд (маркетинг, HR) и IT-команд со смешанными процессами \\
    \hline
    ClickUp & Универсальная (All-in-One) & Широчайший функционал (задачи, документы, цели), гибкая настройка & Команд, желающих объединить все рабочие процессы в одном инструменте \\
    \hline
    GitLab & Agile, DevOps & Бесшинная интеграция управления задачами с репозиторием и CI/CD & Команд разработки, использующих GitLab для всего цикла DevOps \\
    \hline
    Redmine & Универсальная & Бесплатность (Open Source), полный контроль над данными, расширяемость плагинами & Технически подкованных команд и компаний с высокими требованиями к безопасности \\
    \hline
    \end{tabular}
    \end{table}

\subsection{Выбор инструмента для управления проектом}

Для управления проектом разработки системы распознавания объектов на трамвайных путях выбран гибридный подход Scrumban. Это обусловлено двойственной природой проекта, сочетающей:
\begin{itemize}
    \item Исследовательские задачи с высокой неопределённостью (эксперименты с нейросетевыми архитектурами, подбор гиперпараметров)
    \item Чёткие инженерные задачи (разработка API, интеграция с видеопотоками, создание интерфейсов)
\end{itemize}

Scrumban позволяет сохранить гибкость Kanban для исследовательского блока работ и дисциплину Scrum для разработки. В качестве инструмента выбран Kaiten, обеспечивающий:
\begin{itemize}
    \item Настройку гибридных рабочих процессов
    \item Визуализацию процессов
    \item Гибкое планирование итераций с возможностью оперативного перепланирования
\end{itemize}

Данный подход оптимально соответствует требованиям проекта, обеспечивая баланс между гибкостью и контролем на всех этапах разработки.
\subsection{Обоснование выбора Kaiten}
\begin{itemize}
    \item \textbf{Соответствие требованиям законодательства РФ}. Kaiten обеспечивает полное соблюдение норм 152-ФЗ «О персональных данных», что особенно важно при работе с системами видеонаблюдения на объектах транспортной инфраструктуры.
    
    \item \textbf{Поддержка гибридной методологии Scrumban}. Платформа предоставляет гибкие инструменты для одновременного управления как исследовательскими задачами (через Kanban-доски), так и разработческими процессами (с использованием Scrum-подхода).
    
    \item \textbf{Визуализация сквозных рабочих процессов}. Kaiten позволяет отслеживать полный цикл работ — от экспериментов с архитектурами нейросетей до интеграции готовых модулей в систему, что обеспечивает прозрачность всего проекта.
    
    \item \textbf{Адаптивность под специфику ML-проектов}. Инструмент поддерживает настройку специализированных рабочих процессов для задач компьютерного зрения, включая управление датасетами, экспериментов с моделями и валидации результатов.
    
    \item \textbf{Техническая поддержка и развитие}. Как российский продукт, Kaiten гарантирует бесперебойную работу и постоянное развитие функционала без рисков санкционных ограничений.
\end{itemize}

Kaiten демонстрирует оптимальное сочетание функциональности для гибкого управления исследовательскими проектами и соответствия требованиям российского законодательства, что делает его наиболее подходящим выбором для реализации системы распознавания объектов на трамвайных путях.

\newpage
\section{Проектирование и управление разработкой}

На основе проведённого анализа методологий и инструментов управления проектами, данная глава описывает практическое применение выбранных подходов для реализации проекта системы распознавания посторонних объектов на трамвайных путях. 

\textbf{Цель работы:} Организовать процесс управления разработкой системы компьютерного зрения для детекции объектов в реальном времени, используя гибридную методологию Scrumban и отечественное программное обеспечение Kaiten.

\textbf{Ключевые этапы проекта:}

\begin{enumerate}
    \item \textbf{Инициация и планирование}
    \begin{itemize}
        \item Формализация технических требований к системе
        \item Формирование проектной команды (Data Scientist, ML-инженер, разработчик)
        \item Настройка рабочего пространства в Kaiten
        \item Создание первичного бэклога продукта
    \end{itemize}
    
    \item \textbf{Подготовка данных и инфраструктуры}
    \begin{itemize}
        \item Сбор и разметка датасета изображений трамвайных путей
        \item Разработка пайплайна предобработки изображений
        \item Настройка вычислительной инфраструктуры для обучения моделей
        \item Развертывание среды для экспериментов
    \end{itemize}
    
    \item \textbf{Разработка MVP системы}
    \begin{itemize}
        \item Эксперименты с архитектурами нейросетей (YOLO, SSD)
        \item Реализация модуля детекции в реальном времени
        \item Разработка механизма фильтрации ложных срабатываний
        \item Создание API для интеграции с системами видеонаблюдения
    \end{itemize}
    
    \item \textbf{Итерационное развитие системы}
    \begin{itemize}
        \item Проведение циклических экспериментов по улучшению точности
        \item Оптимизация производительности для работы в реальном времени
        \item Доработка функционала на основе тестовых эксплуатаций
        \item Расширение классов детектируемых объектов
    \end{itemize}
    
    \item \textbf{Завершение проекта}
    \begin{itemize}
        \item Финальное тестирование и валидация системы
        \item Подготовка эксплуатационной документации
        \item Обучение персонала и передача системы в эксплуатацию
    \end{itemize}
\end{enumerate}

\subsection{Обоснование выбора метода управления}

Выбор гибридной методологии Scrumban обусловлен специфическими особенностями проекта компьютерного зрения:

\begin{enumerate}
    \item \textbf{Высокая неопределённость результатов исследований}. Эффективность различных архитектур нейросетей и методов обработки изображений невозможно точно предсказать на этапе планирования.
    
    \item \textbf{Комбинация исследовательских и инженерных задач}. Проект совмещает эксперименты с моделями (требующие гибкости Kanban) и разработку стабильных компонентов (эффективно управляемых через Scrum).
    
    \item \textbf{Необходимость быстрой адаптации к результатам экспериментов}. Полученные метрики качества детекции требуют оперативного пересмотра приоритетов и подходов.
    
    \item \textbf{Требования к работе в реальном времени}. Жёсткие ограничения по производительности системы требуют итеративной оптимизации и тестирования.
\end{enumerate}

\textbf{Преимущества Scrumban для данного проекта:}

\begin{itemize}
    \item Гибкое управление потоком исследовательских задач через Kanban-доски
    \item Структурированная разработка стабильных компонентов через спринты
    \item Возможность оперативного перераспределения ресурсов между исследованиями и разработкой
    \item Постоянная адаптация процессов на основе метрик качества модели и производительности системы
\end{itemize}

Данный подход позволяет эффективно сочетать исследовательскую гибкость с инженерной дисциплиной, что критически важно для успешной реализации проекта систем компьютерного зрения.

\subsection{Использование программного средства}
Kaiten представляет собой облачную платформу для управления проектами, доступную через веб-интерфейс, что обеспечивает возможность коллективной работы распределённой команды.
Для управления проектом системы распознавания объектов было развернуто единое рабочее пространство, реализующее принципы гибридной методологии Scrumban.

Начальная доска представлениа на рисунке \ref{fig:firstpage}

\begin{figure}[H]
    \centering
    \includegraphics[width=0.78\linewidth]{img/firstpage.png}
    \caption{Начальная доска}
    \label{fig:firstpage}
\end{figure}

Для описание первой задачи, необходимо перейти в "списки", и добавить в очередь задачу (см. рис. \ref{fig:tolist}).

\begin{figure}[H]
    \centering
    \includegraphics[width=0.78\linewidth]{img/tolist.png}
    \caption{Списки}
    \label{fig:tolist}
\end{figure}

Для добавления задачи, необходимо записать ее в очередь задач как представлено на рисунке \ref{fig:formtask.png}.

\begin{figure}[H]
    \centering
    \includegraphics[width=0.78\linewidth]{img/formtask.png}
    \caption{Добавление задачи}
    \label{fig:formtask}
\end{figure}

Задачу в бэклогах задает ответственный за проект. Так же можно добавлять комментарии внутри задач (рисунок \ref{fig:firsttask}).

\begin{figure}[H]
    \centering
    \includegraphics[width=0.78\linewidth]{img/firsttask.png}
    \caption{Задача и комментарии}
    \label{fig:firsttask}
\end{figure}

После добавления задач проекта, можно увидить все задачи в бэклоге продукта (рисунок \ref{fig:backlog}).

\begin{figure}[H]
    \centering
    \includegraphics[width=0.48\linewidth]{img/backlog.png}
    \caption{Беклог}
    \label{fig:backlog}
\end{figure}

В самих задачах, можно устанавливать сроки выполнения задач, так же есть возможность почемачать срочные задачи (представлено на рисунке \ref{fig:timer}).

\begin{figure}[H]
    \centering
    \includegraphics[width=0.78\linewidth]{img/timer.png}
    \caption{Установка сроков}
    \label{fig:timer}
\end{figure}

Помимо таймеров, можно так же устанавливать:
\begin{itemize}
    \item метки; 
    \item чек-лист;
    \item файлы;
    \item родительская карточка;
    \item дочерняя карточка;
    \item трудозатраты;
    \item размер;
    \item тип.
\end{itemize}

Так же на домашней странице проекта есть раздел "спринтов", в ней можно создавать отдельные задачи, которые имеют приоритет, находятся в работе, тестируются или готовы. Представлено на рисунке \ref{fig:sprint}
\begin{figure}[H]
    \centering
    \includegraphics[width=0.78\linewidth]{img/sprint.png}
    \caption{Спринты}
    \label{fig:sprint}
\end{figure}

В окне "таблица", можно увидеть все задачи, ответственных за них и участников. При этом каждой задаче можно ставить свои тэги для визуального упращения (рисунок \ref{fig:table}).

\begin{figure}[H]
    \centering
    \includegraphics[width=0.78\linewidth]{img/table.png}
    \caption{Таблица в Kaiten}
    \label{fig:table}
\end{figure}

Далее с рисунка \ref{fig:startwork} по \ref{fig:endwork}, показан пример полноценной работы с задачей. 

\begin{figure}[H]
    \centering
    \includegraphics[width=0.78\linewidth]{img/startwork.png}
    \caption{Пример с описанием и комментарием}
    \label{fig:startwork}
\end{figure}

\begin{figure}[H]
    \centering
    \includegraphics[width=0.78\linewidth]{img/bug.png}
    \caption{Пример с помечанием ошибки}
    \label{fig:bug}
\end{figure}

\begin{figure}[H]
    \centering
    \includegraphics[width=0.78\linewidth]{img/endwork.png}
    \caption{Пример с помечанием нововведения в проект}
    \label{fig:endwork}
\end{figure}

В Kaiten так же можно просматривать таймлайны по картачкам в секции "timeline" (показано на рисунке \ref{fig:timeline})

\begin{figure}[H]
    \centering
    \includegraphics[width=0.78\linewidth]{img/timeline.png}
    \caption{Таймлайн проекта}
    \label{fig:timeline}
\end{figure}

Для ведения аналитики и отчетности, в Kaiten можно использовать секцию "отчетность", где можно выбрать несколько видов отчетности (рисунок \ref{fig:analit})

\begin{figure}[H]
    \centering
    \includegraphics[width=0.78\linewidth]{img/analit.png}
    \caption{Отчетность}
    \label{fig:analit}
\end{figure}

В "отчетности" так же доступны:

\begin{itemize}
    \item Накопительная диаграмма потока;
    \item Суммарный отчет;
    \item Время цикла;
    \item Спектральная диаграмма;
    \item Динамика изменения времени цикла;
    \item Пропускная способность;
    \item Распределение карточек;
    \item Скорость выполнения;
    \item Спринты;
    \item Контрольный график.
\end{itemize}

Ниже на рисунке \ref{fig:diag}, приведен пример накопительной диаграммы:
\begin{figure}[H]
    \centering
    \includegraphics[width=0.78\linewidth]{img/diag.png}
    \caption{Накопительная диаграмма потока}
    \label{fig:diag}
\end{figure}


Выбор программного обеспечения Kaiten для управления проектом системы распознавания объектов полностью обоснован его соответствием методологии Scrumban.

Kaiten обеспечивает комплексную поддержку гибридного подхода Scrumban через:

\begin{itemize}
    \item \textbf{Гибкие рабочие процессы} — возможность настройки различных типов досок (Kanban для исследовательских задач и Scrum для разработки) в рамках единого рабочего пространства;
    \item \textbf{Визуализация потоков работ} — отображение полного цикла задач от формулирования гипотезы до внедрения в эксплуатацию;
    \item \textbf{Адаптивное планирование} — инструменты для итеративного планирования спринтов с возможностью оперативного перераспределения ресурсов.
\end{itemize}

Для специфических требований проекта компьютерного зрения Kaiten предоставляет:

\begin{itemize}
    \item \textbf{Управление исследовательскими процессами} — специализированные доски для экспериментов с архитектурами нейросетей, обработки данных и валидации моделей;
    \item \textbf{Аналитика и метрики} — встроенные инструменты для отслеживания прогресса и качества выполнения задач.
\end{itemize}

\newpage
\section*{Заключение}
\addcontentsline{toc}{section}{Заключение}

В рамках данной работы была успешно решена задача выбора, обоснования и практического описания применения методологии и инструментального средства управления для проекта по разработке системы распознавания посторонних объектов на трамвайных путях в режиме реального времени. Был проведен системный анализ существующих подходов к управлению проектами, включая классические, гибкие и гибридные методологии, а также выполнен обзор современного рынка программных средств управления проектами.

На основе анализа специфики проекта, характеризующегося высокой степенью неопределенности, наличием научно-исследовательского компонента в области компьютерного зрения и необходимостью сочетания экспериментальных работ с инженерной разработкой, был сделан вывод об оптимальности применения гибридной методологии Scrumban. Данный подход позволяет эффективно комбинировать гибкость Kanban для управления исследовательскими задачами и дисциплину Scrum для разработки стабильных компонентов системы.

В качестве инструментального средства, с учетом требований законодательства РФ в области обработки данных и необходимости визуализации сложных рабочих процессов, был выбран отечественный продукт Kaiten. Практическая часть работы наглядно продемонстрировала, что его функционал полностью обеспечивает реализацию гибридной методологии Scrumban. Было показано, как с помощью системы досок, бэклога продукта, спринтов и аналитических отчетов организуется полный цикл управления проектом.

Ключевым результатом работы стало проектирование специализированной структуры рабочего пространства в Kaiten, адаптированной под специфику проекта компьютерного зрения. Организация раздельных досок для исследовательских задач, разработки и управления данными позволяет эффективно распределять ресурсы команды и обеспечивает прозрачность всех процессов.

Таким образом, цель работы достигнута. Выбранная связка гибридной методологии Scrumban и инструмента Kaiten представляет собой сбалансированное решение, полностью адаптированное для успешной реализации сложных проектов в области машинного обучения и компьютерного зрения, сочетающих исследовательскую деятельность и инженерную разработку.

\newpage
\addcontentsline{toc}{section}{Список литературы}
\begin{thebibliography}{9}
    \bibitem{kaiten}
    Kaiten – система управления проектами и процессами. [Электронный ресурс]. URL: https://kaiten.ru (дата обращения: 12.11.2025).
    
    \bibitem{agile}
    Beck K. et al. Manifesto for Agile Software Development // Agile Manifesto. – 2001. [Электронный ресурс]. URL: https://agilemanifesto.org/iso/ru/manifesto.html (дата обращения: 12.11.2025).
    
    \bibitem{pmbok}
    Институт управления проектами. Руководство к Своду знаний по управлению проектами (Руководство PMBOK®) – Седьмое издание. – Newtown Square, PA: Project Management Institute, Inc., 2021. – 370 с.
    
    \bibitem{jira}
    Atlassian Jira – инструмент для управления проектами в Agile. [Электронный ресурс]. URL: https://www.atlassian.com/ru/software/jira (дата обращения: 12.11.2025).
    
    \bibitem{kanban}
    Андерсон Д. Канбан: Альтернативный путь в Agile. – М.: Манн, Иванов и Фербер, 2017. – 352 с.
    
    \bibitem{scrum}
    Швабер К., Сазерленд Д. Руководство по Scrum. Официальное определение Scrum. – Scrum.org, 2020. – 19 с.
    
    \bibitem{rag}
    Lewis P., Perez E., et al. Retrieval-Augmented Generation for Knowledge-Intensive NLP Tasks // Advances in Neural Information Processing Systems (NeurIPS). – 2020. – Vol. 33. – P. 9459-9474.
    \end{thebibliography}

\end{document}
