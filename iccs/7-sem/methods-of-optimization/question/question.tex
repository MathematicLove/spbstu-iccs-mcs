\documentclass[areasetadvanced]{scrartcl}

\usepackage[utf8]{inputenc}
\usepackage[T2A]{fontenc}
\usepackage[english,russian]{babel}

\usepackage[footskip=1cm,left=25mm, right=15mm, top=20mm, bottom=20mm]{geometry}
\usepackage{setspace}
\usepackage{amsmath, amssymb} 
\usepackage{graphicx}
\usepackage{tikz}
\usetikzlibrary{arrows.meta}
\usepackage{float}
\usepackage{dashrule}
\usepackage{fancyhdr} 
\usepackage{hyperref} 
\usepackage{parskip}
\usepackage{textcomp, enumitem}
\usepackage{indentfirst}
\usepackage{graphicx}
\usepackage{algorithm}
\usepackage{algpseudocode}
\usepackage{array} 
\usepackage{geometry}
\usepackage{afterpage}
\usepackage{minted}
\setcounter{secnumdepth}{3} 
\setcounter{tocdepth}{3}    
\usepackage{listings} 

\newcommand{\icon}[1]{\includegraphics[height=1.2em]{#1}}

\tikzstyle{block} = [rectangle, rounded corners, minimum width=3cm, minimum height=1cm, text centered, draw=black, fill=lightgray]

\setkomafont{sectioning}{\normalfont\bfseries} 
\setkomafont{section}{\normalfont\Large\bfseries}
\setkomafont{subsection}{\normalfont\large\bfseries}
\setkomafont{subsubsection}{\normalfont\large\bfseries}
\setkomafont{paragraph}{\normalfont\large\bfseries} 

\lstset{
  language=Haskell,
  basicstyle=\ttfamily\small,
  keywordstyle=\color{blue}\bfseries,
  stringstyle=\color{red},
  commentstyle=\color{green!70!black},
  numbers=left,
  numberstyle=\tiny,
  stepnumber=1,
  numbersep=10pt,
  showstringspaces=false,
  breaklines=true,
  frame=single
}

\lstdefinelanguage{Lua}{
    keywords={function, end, if, then, else, elseif, for, while, do, repeat, until, break, return, local, and, or, not, true, false, nil},
    keywordstyle=\color{blue}\bfseries,
    stringstyle=\color{red},
    commentstyle=\color{green!70!black},
    morestring=[s]{"}{"},
    morestring=[s]{'}{'},
    morecomment=[l]{--},
    morecomment=[s]{--[[}{]]},
    basicstyle=\ttfamily\small,
    numbers=left,
    numberstyle=\tiny,
    stepnumber=1,
    numbersep=10pt,
    showstringspaces=false,
    breaklines=true,
    frame=single
}

\setlength{\parindent}{1.25cm}
\setcounter{tocdepth}{2}
\begin{document}
\sloppy
	\thispagestyle{empty}
    \begin{center}
        \Huge {Методы оптимизации} 
        \vspace{5.5cm}

        \huge {\guillemotleft Дихотомия и задача об укладке рюкзака\guillemotright}

        \vspace{0.5cm}
		\large \textbf{{Салимли Айзек, гр. 5130201/20101}}
    \end{center}
	\vfill
	\vspace{14.5cm}
	\begin{flushright}
		\guillemotleft \rule[0pt]{0.8cm}{0.5pt}\guillemotright \rule[0pt]{2cm}{0.5pt} 20\rule[0pt]{0.5cm}{0.5pt} г.
	\end{flushright}
	\vfill
	\begin{center}
		Санкт-Петербург, 2025
	\end{center}
	\newpage
    \section{Задача о рюкзаке}
    Задача о рюкзаке - задача комбинаторной оптимизации и теории алгоритмов. 

    Проблема состоит в том, что необходимо принимать дискретное решение при наличии непрерывных ограничений. 
    Такое противоречие порождает дихотомию. 

    Математическая модель задачи: 
$$
    max \sum_{i=1}^{n}v_i x_i
$$
при условиях: 
$$
    \sum_{i = 1}^{n} w_i x_i \leq W, x_i \in {0,1}
$$

где: $v_i$ - ценность объекта, $w_i$ - объем или вес объекта, $W$ - вместимость рюкзака, $x_i$ - бинарная переменная включения объекта. 

Сущность дихотомии заключается в следующем: 

\begin{enumerate}
    \item Ограничения на ресурс (вес, объем, пространство) - является непрерывным 
    \item Решение (включать / не включать объект) - дискретное 
\end{enumerate}
это приводит к тому, что оптимальное решение редко полностью заполняет доступное пространство и возникает остаточное (свободное) пространство,
которое невозможно эффективно использовать из-за дискретности объекта.

То есть дихотомия задачи заключается между аддитивной модели ограничения и неделимостью объектов.

Формально $W \in \mathbb{R}_+, w_i \in \mathbb{R}_+, \texttt{ a } x_i \in {0,1}$.

$\Rightarrow$ останеться пространство.

\section{Оценка свободного пространства и связь с задачей} 

Свободное пространство - разница между вместимостью рюкзака и суммарным объемом уложеных объектов.

$$
    F_s = W - \sum_{i = 1}^{n} w_i x_i
$$

Из-за проблемы из пункта 1, даже при оптимальном выборе объектов, $F_s > 0$.

\section{Случай при N = 3}
Для задач в трехмерном пространстве (например упаковка параллепипедов в контейнер) доказана, что существует верхняя оценка доли 
свободного пространства независящая от конфигурации объектов, а зависит только от размерности N.

Связь с задачей о рюкзаке в том, что каждый объект имеет фиксированный объем, контейнер (рюкзак) имеет ограниченный объем и дискретность
объектов приводит к остаткам пространства. 

\section{Хранение данных N = 8, 24}
В структурах хранения данных такие как: RAID-массивы, распределенные файловые системы, блоковое хранение в UNIX системах. 
Данные разбиваются на блоки фиксированного размера, при этом размер файлов не кратен размеру блока.

N = 8, 24 - соответствует размерности адресного пространства/ структурного пространства (число бит в адресном блоке или размер блока).

Пример: 
1 байт = 8 бит, адресное пространство кратно байту, минимальный размер блока хранения - 1 байт либо ему кратные. 
Но если файл имеет размер некратный 8 битам, то возникает внутреняя фрагментация и часть блока остается незаполненной. 

Аналогично каждый файл - предмет
Блок - ячейка рюкзака 
Свободные байты - свободное пространство.

N = 24 - в более сложных системах: RGB-цвет, сетевые адреса, хеш-пространства. 

\section{Связть с линейным программированием (LP)}
Если заменить условие $x_i \in {0,1}$ на: 

$$
    0 \leq x_i \leq 1 
$$

то мы получим линейную релаксацию и задача становиться задачей LP. 

Тогда: оптимальное решение полностью заполняет рюкзак $\Rightarrow F_s \geq 0$

при этом решение находится жадным алгоритмом. 

Разница между оптимумом линейной релаксации и целочисленной задачей называется - интегральный разрыв.

Этот разрыв отображает количественно: 
\begin{enumerate}
    \item Величину неизбежного свободного пространства
    \item Последствие дихотомии (непрерывное - дискретное)
\end{enumerate}

\section{Связь задачи о рюкзаке и заполнение пространства}
Задача о рюкзаке - одномерный случай более общей задачей заполнения пространства. 

То есть: 
\begin{itemize}
    \item 1D - задача о рюкзаке 
    \item 2D - упаковка прямоугольников
    \item 3D - упаковка параллепипедов
    \item N - мерное - размещение в абстрактных пространствах
\end{itemize}

\section{Пример заполнения пространства сферами}

% КУБ 
\begin{figure}[H]
    \center
    \includegraphics[width=0.23\linewidth]{img/Square_circle_grid_spheres.png}
    \caption{Упаковка сфер}
\end{figure}

\textbf{Задача:} 

Максимально плотная упаковка одинаковых сфер радиуса $r$ в куб объемом $V$

контейнер - трехмерный куб $C \subset \mathbb{R}^3$

объекты - сферы объема $V_s = \frac{4}{3} \pi r^3$

\textbf{Ограничения: }

\begin{itemize}
    \item Сферы нельзя пересикать,
    \item Нужно максимизировать число сфер n.
\end{itemize}

\textbf{Аналогия с задачей о рюкзаке: }
\begin{itemize}
    \item Каждая сфера - дискретный объект фиксированного объема
    \item Куб - контейнер
\end{itemize}

Свободное пространство возникает так как: 

Сферы не могут идеально заполнить евклидово пространство.

максимальная плотность упаковки сфер в $\mathbb{R}^3$

$$
    \delta_{max} = \frac{\pi}{\sqrt{18}} \approx 0.74
$$

$\Rightarrow$ не менее 26\% объема пустые даже в оптимальном случае. 



Плотность упаковки определяется как: 

$$
\delta = \lim_{R \rightarrow +\infty} \frac{\texttt{Объем сфер внутри шара радиусом R}}{\texttt{Объем шара радиусом R}}
$$

$\delta$ - показывает какую долю пространства можно заполнить сферами без пересечений.

a - ребро куба 
в первой ячейки 4 сферы.

Сферы касаются друг друга по диоганалям граний $a \sqrt{2} = d$

По этой диоганали укладывается 4 радиуса $r$. 

$$
    a \sqrt{2} = 4r \Rightarrow a = 2\sqrt{2} r 
$$

$$
    V = a^3 = (2\sqrt(2) r)^3 = 16 \sqrt{2} r^3
$$

4 сферы: 

$$
    V_{4s} = \frac{16}{3} \pi r^3
$$

$$
    \Rightarrow \delta = \frac{V_{4s}}{V} = \frac{16 \pi r^3}{16\sqrt{2} r^3} = \frac{\pi}{\sqrt{2}} = \frac{\pi}{\sqrt{18}}
$$

$$
    \Rightarrow \delta_{max} = \frac{\pi}{\sqrt{18}} \approx 0.74
$$

Томас Хейлс доказал гипотизу Кеплера, что ни одна упаковка сфер в $\mathbb{R}^3$, не может иметь плотность выше $\frac{\pi}{\sqrt{18}}$.

\section{Практические примеры}

\begin{itemize}
    \item Распределение памяти и дискового пространства 
    \item VRP - задачи
    \item Планирование ресурсов в вычислительных системах
\end{itemize}

Но оптимизация выбора объектов не гарантирует оптимального использования пространства.
\end{document}